
\questionheader{\theproblemnumber\ Jogging (2)}

\Question\ Sarah is about to take up jogging. She will start jogging $D$ metres per day
and increase her daily distance by $I$ metres each day. When she reaches her target of $T$
metres per day, she will stop the daily increase and just keep a consistent jogging
pattern.

Find the total distance she jogs in $N$ days.

\Input

The input contains four lines.
\begin{itemize}
  \item $D$, the distance (m) at which she starts jogging;
  \item $I$, the amount by which she increases her jog each day (m);
  \item $T$, her target daily jog (m), after which she doesn't increase it any further.
  \item $N$, the number of days for which we will calculate the total distance.
\end{itemize}

All values are integers, and you are guaranteed that $D < D+I < T$, that is, she will have
to apply at least two increases before reaching her target.

\Output\ You will write a single integer to \OUT: the total distance she jogs in $N$ days.

\Sample

\minipagestwo{%
  \sample{0.4}{100\\50\\2000\\6}
         {0.4}{1350} }{%
  \sample{0.4}{700\\20\\750\\6}
         {0.4}{4410} }


\Explanation In the first sample, over the first six days, Sarah runs a total of
$100+150+200+250+300+350=1350$ metres.

In the second sample, over the first six days, Sarah runs a total of
$700+720+740+750+750+750=4410$ metres.

\Scratch\ It is intended here that you sum the distances in a loop. It is possible to work
these out using mathematical formulas, but that is: (a) complicated; and (b) counter to
the spirit of this chapter.

