\questionheader{Shopping (3)}

\Note\ This is the same as \emph{Shopping (2)} but with the data presented differently:
all the quantities are grouped and all the prices are grouped. This makes it more
difficult because you can't just calculate the total as you go; you need to store the
values in lists first.

\Question\ You are in the middle of a sizeable shopping trip and realise you may not have
enough money to pay for everything you have put in your trolley. Quickly, you write a
program that takes in the quantities and prices of all the items you intend to purchase
and reports the total price.

\Input

The first line of input is $N$, which is the number of quantity-item pairs. Following this
are $2N$ lines: $N$ lines of quantities and $N$ lines of their respective prices.

\begin{itemize}
  \item $0 < N < 10\,000$.
  \item Each quantity $Q_i$ is an integer that satisfies $0 < Q_i < 1000$.
  \item Each price $P_i$ is a float that satisfies $0.00 < P_i < 1000.00$.
\end{itemize}

\Output\ The output is a single float containing the total price, and it must be rounded
to two decimal places.%

\Sample

\sample{0.2}{5\\4\\1\\3\\19\\7\\2.99\\3.15\\14.95\\0.14\\7.10}
       {0.2}{45.01}

\Explanation The output represents \$97.37, which is $4(\$2.99) + 1(\$3.15) + 2(\$14.95)
+ 19(\$0.14) + 7(\$7.10)$.

