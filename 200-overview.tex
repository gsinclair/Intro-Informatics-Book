
\chapter{Looping through data}

In the problems you've encountered so far, you would:
\begin{itemize}
  \item read a fixed number of items from \IN;
  \item perform some calculation on those items;
  \item write a result to \OUT.
\end{itemize}

I'm sure you understand, though, that most problems don't come with a fixed number of
items of input, and sooner or later you will need to write code that can handle input of
any size. Well, that time is now.



% ---------------------------------------------------------------- 201 Tallest (3)

\clearpage

\questionheader{Who is the tallest? (3) \workedexample}

\Question You run a photography studio for sporting teams. Before a team comes in for a
photo shoot, you ask them to submit the height of each person in centimetres. This helps
you to arrange studio props appropriately. In particular, you want to know the height of
the tallest person in the photo so that the artwork on the wall can be placed in the best
position.

\Input A positive integer $N$, followed by $N$ lines, each containing a person's height in
centimetres.

\Sample

\minipagestwo{\sample{0.3}{8\\165\\177\\172\\180\\175\\179\\181\\180}{0.3}{181}}
             {\sample{0.3}{5\\127\\128\\128\\128\\127}{0.3}{128}}

\Explanation In the first sample, there are eight heights given and the greatest of these
is \texttt{181}. In the second sample, there are five heights given and the greatest of
these is \texttt{128}. The fact that the ``greatest'' height occurred three times does not
matter.

\Scratch\ In all problems so far, we have known exactly how much data to read. Here we
read the first line and it \emph{tells} us how much data to read! This means we need a
program loop.

Loops are discussed in more detail in \futurereference{Chapter 3}, but we only need a very
basic use of them here: to repeat an action $N$ times.

The beginning of our solution looks like (excluding the \pycode|ex201| function header):
\begin{pythoncode}
  N = int(IN.readline())
  for i in range(N):
    height = int(readline())
    # ... do something with height ...
\end{pythoncode}

It is hopefully clear that if the first line of input is \texttt{8}, then the code above
will read that value and then read eight more lines.

So what to do with each new height that we read? We are trying to find the maximum height,
so it would be good to keep track of that as we go. We start off initialising it to zero:
\begin{pythoncode}
  maxheight = 0
\end{pythoncode}

Within the loop, each new height we read is \emph{potentially} the maximum height. How do
we know whether it is? Well, if it's greater than our current maximum, of course!
\begin{pythoncode}
  if height > maxheight:
    # ...
\end{pythoncode}

And if it \emph{is} greater than our current maximum, then we update our current maximum.
\begin{pythoncode}
  if height > maxheight:
    maxheight = height
\end{pythoncode}

This means that every time the loop runs, the \pycode|maxheight| variable contains the
\emph{greatest height seen so far}. And therefore, when the loop has run its prescribed
number of times, \pycode|maxheight| contains the greatest height overall.

It's pretty simple, really, but you need to internalise every little detail of this
solution and adapt the techniques to the coming problems.

\Solution

Putting all the above together, we get:

\begin{pythoncode} 
  def ex201(IN, OUT):
    N = int(IN.readline())
    maxheight = 0

    for i in range(N):
      height = int(IN.readline())
      if height > maxheight:
        maxheight = height

    print(maxheight, file=OUT)
\end{pythoncode}

\Notes\ For now, we treat the code \pycode|for i in range(N)| as a magic incantation
telling Python to do something $N$ times. The \emph{loop variable} $i$ is being completely
ignored in this code, because we don't care how many times we've been through the loop so
far. But we need to have a loop variable, so $i$ will do.

It is tempting to shorten the name \pycode|maxheight| to \pycode|max|. But as you can see
from the way \pycode|max| is coloured, we shouldn't do that: \pycode|max| is a special
word in Python that we don't want to interfere with.

It is worth looking at a \emph{trace table} of this algorithm. Sketching such a table
\emph{before} you write any code is a valuable skill in solving problems. We are tracing
the data used in the first sample. The $i$ column is included to show the looping but
rendered in grey because we don't actually use this variable.

\begin{center}
  \begin{tabular}{lll}
    \toprule
    \emph{i\quad} & \emph{height} & \emph{maxheight} \\
    \midrule
    -        & -             & 0                \\
    \grey{0} & 165           & 165              \\
    \grey{1} & 177           & 177              \\
    \grey{2} & 172           &                  \\
    \grey{3} & 180           & 180              \\
    \grey{4} & 175           &                  \\
    \grey{5} & 179           &                  \\
    \grey{6} & 181           & 181              \\
    \grey{7} & 180           &                  \\
    \bottomrule
  \end{tabular}
\end{center}

The \emph{maxheight} column shows exactly when that variable is updated, which occurs in
Line~8.
