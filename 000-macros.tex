
% -------------- Problem number counter, chapter, question header

\newcounter{problemnumber}

\newcommand{\chaptermacro}[1]{\chapter*{#1}
  \addcontentsline{toc}{chapter}{#1}}

\newcommand{\questionheader}[1]{\stepcounter{problemnumber}
  \section*{\theproblemnumber\quad #1}
  \addcontentsline{toc}{section}{\theproblemnumber\hspace{0.5em} #1}}

\newcommand{\questionheadercontinued}[1]{{\Large\color{RoyalPurple}
  \theproblemnumber\quad #1\quad\textit{continued\dots}
  \vspace{18pt}
}}

% -------------- IN, OUT, colours, future reference, note

\newcommand{\IN}{\texttt{IN}}
\newcommand{\OUT}{\texttt{OUT}}

\definecolor{MyGrey}{gray}{0.45}
\newcommand{\grey}[1]{{\color{MyGrey}#1}}

\newcommand{\futurereference}[1]{\textbf{\color{Plum}Ref: #1}}
\newcommand{\note}[1]{\textbf{\color{Orchid}Note: #1}}

% -------------- Question, Input, Output, Sample, Explanation, ......

\newcommand{\Question}{\textbf{\color{purple}Question}\quad}
\newcommand{\Note}{\vspace{6pt}\textbf{\color{purple}Note}\quad}
\newcommand{\Input}{\vspace{6pt}\textbf{\color{purple}Input}\quad}
\newcommand{\Output}{\vspace{6pt}\textbf{\color{purple}Output}\quad}
\newcommand{\Sample}{\vspace{6pt}\textbf{\color{purple}Sample}\quad}
\newcommand{\Explanation}{\vspace{6pt}\textbf{\color{purple}Explanation}\quad}
\newcommand{\Scratch}{\vspace{6pt}\textbf{\color{purple}Scratch}\quad}
\newcommand{\Solution}{\vspace{6pt}\textbf{\color{purple}Solution}\quad}
\newcommand{\Notes}{\vspace{6pt}\textbf{\color{purple}Notes}\quad}
\newcommand{\Afterword}{\vspace{6pt}\textbf{\color{purple}Afterword}\quad}
\newcommand{\Running}{\vspace{6pt}\textbf{\color{purple}Running the code}\quad}
\newcommand{\Hint}{\vspace{6pt}\textbf{\color{purple}Hint}\quad}

% -------------- workedexample, stronghint

\newcommand{\workedexample}{{\color{purple}(worked example)}}
\newcommand{\stronghint}{{\color{purple}(strongly hinted)}}

% -------------- Python code listings

\newminted{python}{xleftmargin=2cm, linenos}
\newmintinline[pycode]{python}{}

% -------------- \sample and its encompassing minipagestwo and minipagesthree

%% % <hack> Frame all minipages, to aid in layout. Comment out when not wanted,
%% %        which is most of the time.
%% \let\minipagebak\minipage
%% \let\endminipagebak\endminipage
%% \newsavebox\TestBox
%% \renewenvironment{minipage}[2][]
%% {\begin{lrbox}{\TestBox}\begin{minipagebak}[#1]{#2}}
%% {\end{minipagebak}\end{lrbox}\fbox{\usebox{\TestBox}}}
%% % </hack>

\newcommand{\sample}[4]{%
  \begin{minipage}[t]{1.2cm}
    ~~~
  \end{minipage} %
  \begin{minipage}[t]{#1\textwidth}
    \textbf{IN}\\[3pt]
    \ttfamily
    #2
  \end{minipage} %
  \begin{minipage}[t]{#3\textwidth}
    \textbf{OUT}\\[3pt]
    \ttfamily
    #4
  \end{minipage}}

\newcommand{\minipagestwo}[2]{
  \begin{minipage}[t]{0.5\textwidth}
    #1
  \end{minipage} %
  \begin{minipage}[t]{0.5\textwidth}
    #2
  \end{minipage}}

\newcommand{\minipagesthree}[3]{
  \begin{minipage}[t]{0.33333\textwidth}
    #1
  \end{minipage} %
  \begin{minipage}[t]{0.33333\textwidth}
    #2
  \end{minipage} %
  \begin{minipage}[t]{0.33333\textwidth}
    #3
  \end{minipage}}

% -------------- environment for inline tables
%                  inlinetable  - bigkip above and below
%                  inlinetable* - medskip above, bigskip below

\newenvironment{inlinetable}{
  \bigskip
  \begin{center}
}{
  \end{center}
  \bigskip
}

\newenvironment{inlinetable*}{
  \medskip
  \begin{center}
}{
  \end{center}
  \bigskip
}

