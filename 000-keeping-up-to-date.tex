
\begin{tcolorbox}
  {\color{BrickRed}\large\bfseries How to keep your book, code and data up to date}

  \paragraph{Book} The version number of the book is on the title page (or here: it is
  \BookVersionString). You can look for the latest version at \url{\BitLyBook}.
  When you run \texttt{l.update()}, it will tell you the most recent available version.

  \paragraph{Code} When you have a working \learninformatics\ environment, you can almost
  always update the code by typing \texttt{l.update()}. If there is a problem, you should
  manually replace the contents of \texttt{learninformatics.py} with
  \url{\BitLyCode}.

  \paragraph{Data} The data to support the judging of exercises lives in
  \texttt{DATA.txt}. This is always obtained and updated by typing \texttt{l.update()}.

  \bigskip
  Your exercises live in \texttt{main.py} and will never be affected by
  running an update.
\end{tcolorbox}

\paragraph{A special note for 0.2 users, 6 July 2021} Version 2.0 of the sofware and data
introduced a technology called YAML, which unfortunately does not work out-of-the-box with
replit.com. There might be a bit of manual updating required as I fix this issue.
