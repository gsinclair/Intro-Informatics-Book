\questionheader{\theproblemnumber\ Shopping (2)}

\Question\ You are in the middle of a sizeable shopping trip and realise you may not have
enough money to pay for everything you have put in your trolley. Quickly, you write a
program that takes in the quantities and prices of all the items you intend to purchase
and reports the total price.

\Input

The first line of input is $N$, which is the number of quantity-item pairs. Following this
are $2N$ lines, each pair of which contains the quantity and the price. The example will
make this clear.

\begin{itemize}
  \item $0 < N < 10\,000$.
  \item Each quantity $Q_i$ is an integer that satisfies $0 < Q_i < 1000$.
  \item Each price $P_i$ is a float that satisfies $0.00 < P_i < 1000.00$.
\end{itemize}

\Output\ The output is a single float containing the total price, and it must be rounded
to two decimal places.

\Sample

\sample{0.2}{5\\4\\2.99\\1\\3.15\\2\\14.95\\19\\0.14\\7\\7.10}
       {0.2}{97.37}

\Explanation The output represents \$97.37, which is $4(\$2.99) + 1(\$3.15) + 2(\$14.95)
+ 19(\$0.14) + 7(\$7.10)$.

