\questionheader{Shopping (2)}

\Question\ You are in the middle of a sizeable shopping trip and realise you may not have
enough money to pay for everything you have put in your trolley. Quickly, you write a
program that takes in the quantities and prices of all the items you intend to purchase
and reports the total price.

\Input

The first line of input is $N$, which is the number of quantity-item pairs. Following this
are $2N$ lines, each pair of which contains the quantity and the price. The example will
make this clear.

\begin{itemize}
  \item $0 < N < 10\,000$.
  \item Each quantity $Q_i$ is an integer that satisfies $0 < Q_i < 1000$.
  \item Each price $P_i$ is a float that satisfies $0.00 < P_i < 1000.00$.
\end{itemize}

\Output\ The output is a single float containing the total price, and it must be rounded
to two decimal places.

\Sample

\sample{0.2}{5\\4\\2.99\\1\\3.15\\2\\14.95\\19\\0.14\\7\\7.10}
       {0.2}{97.37}

\Explanation The output represents \$97.37, which is $4(\$2.99) + 1(\$3.15) + 2(\$14.95)
+ 19(\$0.14) + 7(\$7.10)$.

\Scratch\ After reading the integer $N$, we use it as we did in the last exercise to drive
the loop:
\begin{pythoncode}
  for i in range(N):
    # ...
\end{pythoncode}

The difference here is that we read \emph{two} values each time through the loop, so the
code will look like this:
\begin{pythoncode}
  for i in range(N):
    qty = int(IN.readline())
    pri = float(IN.readline())
    # ...
\end{pythoncode}

Hopefully you can work out what to do to make this a complete solution.

Here is a trace table of the sample data.

\begin{center}
  \begin{tabular}{l l S[table-format=2.2] S[table-format=2.2]}
    \toprule
    \emph{i\quad} & \emph{qty} & \emph{pri} & \emph{total}   \\
    \midrule
                  &            &            & 0.00              \\
    \grey{0}      & 4          & 2.99       & 11.96          \\
    \grey{1}      & 1          & 3.15       & 15.11          \\
    \grey{2}      & 2          & 14.95      & 45.01          \\
    \grey{3}      & 19         & 0.14       & 47.67          \\
    \grey{4}      & 7          & 7.10       & 97.37          \\
    \bottomrule
  \end{tabular}
\end{center}

This problem and the last both demonstrate a common approach to solving the problems in
this chapter:
\begin{itemize}
  \item Decide what your important variables are
  \item Give them initial values
  \item Process each line of input and update the variables
  \item When the processing is complete, the answer is in one of the variables.
\end{itemize}

For finding the tallest person, the important variable was \emph{maxheight}, which was
initialised to zero and updated only when a value is read that exceeds it. For this
shopping scenario, the important variable was \emph{total}, which was intialised to zero
and updated every time according to the newly-read price and quantity.
