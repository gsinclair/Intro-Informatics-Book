
\questionheader{\theproblemnumber\ A fair wage (1)}

\Question\ There are six employees in your company all doing roughly the same kind of
work. Some have worked for you for a longer time, or are generally more productive, so you
pay them a bit more. But once a year you take a look to see if the overall distribution of
wages seems to be fair.

The first thing you look for is the \emph{range} of wages: the highest minus the lowest.
You have a rule that the range should be less than 10\% of the highest wage.

\Input\ From \IN\ you will read six floats, each representing a weekly wage.

\Output\ To \OUT\ you will write three values:
\begin{itemize}
  \item The range, rounded to two decimal places
  \item The highest wage, rounded to two decimal places
  \item \texttt{yes} or \texttt{no} according to whether your wages are fair
\end{itemize}

\Sample

\minipagestwo{\sample{0.4}{535.00\\517.50\\580.00\\575.89\\553.60\\521.45}
                     {0.4}{62.5\\580.0\\no}}
             {\sample{0.4}{535.00\\517.50\\570.00\\570.00\\553.60\\521.45}
                     {0.4}{52.5\\570.0\\yes}}

\Explanation\ In the first sample, the range is about 10.78\% of the maximum wage, thus
the wages are not fair. In the second sample, the range is about 9.21\% of the maximum
wage, thus the wages are fair.

\Scratch\ After completing \emph{Who is the tallest? (2)}, you should be comfortable
reading six numbers and finding their maximum. To find the minimum, just use \pycode/min/
instead of \pycode/max/.

See the first code listing in Chapter 1 if you need a refresher on outputing floats
rounded to two decimal places.

Note that to get the output printed on three lines, you can use three \pycode|print|
statements. Each statement prints its contents on a new line.
\begin{pythoncode}
  print('Line 1')
  print('Line 2')
  print('Line 3')
\end{pythoncode}

Furthermore, note that you need the value of the highest wage \emph{three} times when
completing this exercise: to print it, to calculate the range, and to determine whether
the outcome is fair. Rather than calculating it three times, you should store it in a new
variable:
\begin{pythoncode}
  highest = # ... code to calculate it goes here ...
\end{pythoncode}

Then you can use \pycode|highest| in the remainder of your code.

