\questionheader{Who is the tallest (1)? \stronghint}

\Question\ In the schoolyard you determine who is the tallest by standing back to back.
Among your online friends, though, the only way to find this out is to compare numbers.
There are four people in your group, and so when everyone has entered their height in
centimetres, all that remains is to pick the largest number out of the list.

Your program will read four numbers from \IN\ and write the largest number to \OUT.

\Sample

\sample{0.2}{147\\165\\171\\168}
       {0.2}{171}

\Scratch\ We've seen several examples now of how to read an integer from \IN. For this
problem, we need to do it four times.

\begin{pythoncode}
    a = int(IN.readline())
    b = int(IN.readline())
    c = int(IN.readline())
    d = int(IN.readline())
\end{pythoncode}

We now have all our data. All that remains is to decide which is the largest of $a$, $b$,
$c$ and $d$.

\begin{pythoncode}
    if a >= b and a >= c and a >= d:
        ...
\end{pythoncode}

If the condition above is true, then $a$ is the largest of the four, or at least the equal
largest. What do we do in this scenario?

\begin{pythoncode}
    if a >= b and a >= c and a >= d:
        answer = a
\end{pythoncode}

We store the value of $a$ in a new variable called \emph{answer}.

To the two lines of code provided above, you can add six more to make a complete
determination of the largest value among $a$, $b$, $c$ and $d$.

\Solution\ A complete solution looks like this, except for the six lines you need to add.

\begin{pythoncode}
    def ex103(IN, OUT):
        a = int(IN.readline())
        b = int(IN.readline())
        c = int(IN.readline())
        d = int(IN.readline())

        if a >= b and a >= c and a >= d:
            answer = a
        #
        # ...six more lines...
        #

        print(answer, file=OUT)
\end{pythoncode}

If you're really stuck, you can look at the answers provided.

\Afterword\ The detailed assistance provided in this example was painstaking, so this will
be gradually decreased over the next few problems.

As you coded this problem, you might have had two questions.
\begin{itemize}
    \item Is this \emph{really} the best way to find the largest of four numbers?
    \item What if there were 400 numbers instead of four?
\end{itemize}

These questions will be addressed in \emph{Who is the tallest? (2)} and \emph{Who is the
tallest? (3)} respectively.

