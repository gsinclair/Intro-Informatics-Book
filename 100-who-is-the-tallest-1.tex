\questionheader{\theproblemnumber\ Who is the tallest (1)? \stronghint}

\Question\ In the schoolyard you determine who is the tallest by standing back to back.
Among your online friends, though, the only way to find this out is to compare numbers.
There are four people in your group, and so when everyone has entered their height in
centimetres, all that remains is to pick the largest number out of the list.

Your program will read four numbers from \IN\ and write the largest number to \OUT.

\Sample

\sample{0.2}{147\\165\\171\\168}
       {0.2}{171}

\Scratch\ To read four numbers we just call \code{IN.readline()} four times, storing the
value in a different variable each time. Because we're reading numbers and want to
\emph{treat} them as numbers (not text), we use the \code{int} function as well.

\begin{pythoncode}
    a = int(IN.readline())
    b = int(IN.readline())
    c = int(IN.readline())
    d = int(IN.readline())
\end{pythoncode}

We now have all our data. All that remains is to decide which is the largest of $a$, $b$,
$c$ and $d$.

\begin{pythoncode}
    if a >= b and a >= c and a >= d:
        ...
\end{pythoncode}

If the condition above is true, then $a$ is the largest of the four, or at least the equal
largest. What do we do in this scenario?

\begin{pythoncode}
    if a >= b and a >= c and a >= d:
        answer = a
\end{pythoncode}

We store the value of $a$ in a new variable called \emph{answer}.

To the two lines of code provided above, you can add six more to make a complete
determination of the largest value among $a$, $b$, $c$ and $d$.

\Solution\ A complete solution looks like this, except for the six lines you need to add.

\begin{pythoncode}
    import plcinformatics as p

    def train101(IN, OUT):
        a = int(IN.readline())
        b = int(IN.readline())
        c = int(IN.readline())
        d = int(IN.readline())

        if a >= b and a >= c and a >= d:
            answer = a
        #
        # ...six more lines...
        #

        print(answer, file=OUT)
\end{pythoncode}

If you're really stuck, you can look at the answers provided.

Here are the instructions to run your code.
\begin{itemize}
    \item To run your code interactively: \code{p.run(train101)}, and input four numbers
        using the keyboard, one on each line. The answer will be printed immediately under
        them.
    \item To test it against a small data set: \code{p.test('train101', train101)}. If
        your code has an error, you will get detailed information which might help you to
        fix it.
    \item To judge it against a larger data set: \code{p.judge('train101',
        train101)}.\footnote{For \code{p.test} and \code{p.judge} the \code{'train101'} is
        necessary because the name of the testing or judging data set must be known. For
        \code{p.run} no data set is accessed, so the calling pattern is simpler.} Only
        brief information is given about the success or otherwise of your code.
\end{itemize}

See \futurereference{Appendix ...} for an explanation of the status codes AC, WA, TLE and
RTE.

\Afterword\ The detailed assistance provided in this example was painstaking, so this will
be gradually decreased over the next few problems.

As you coded this problem, you might have had two questions.
\begin{itemize}
    \item Is this \emph{really} the best way to find the largest of four numbers?
    \item What if there were 400 numbers instead of four?
\end{itemize}

The first question will be addressed in (reference: the next problem). The second one will
wait, but not too much longer.

