
\chapter{Operating on a fixed set of numbers}

In these problems, you will read in just one number, or a list of numbers of a known size.
You will calculate something or somethings based on the input and write it to output.

The example problems and solutions shown here demonstrate how to read numbers (integers
and floats) from \IN\ and write results to \OUT. Note that when we write a float%
\footnote{In computing, decimal numbers are generally called ``floats'', short for
    ``floating point numbers'', because the location of the decimal point can float around
    (consider 104.32 and 1.0432, for instance).}
to \OUT, we must round it to the specified number of decimal places to ensure predictable
output.%
\footnote{If you use Python, or pretty much any programming language, to calculate
  $0.1+0.2$ then you will get a surprising result. It's not Python's fault; we have to live
  with the fact that computers, by default, don't handle decimal numbers particularly
well.}

The following lines of code demonstrate how to input and output integers and floats. 

\begin{pythoncode} 
  import learninformatics as l

  def example1(IN, OUT):
    a = int(IN.readline())
    b = float(IN.readline())

    print("a squared:", a*a, file=OUT)
    print("b squared:", b*b, file=OUT)
    print("b squared (2dp):", round(b*b,2), file=OUT)
\end{pythoncode}

You should enter this code, then run it, then type \pycode|l.run(example1)| in your shell.
It will wait while you enter two numbers (an integer and a float) and will then
immediately output the squares that the code promises.

The first two problems in this chapter are complete worked examples to show input and
output in some more context and to demonstrate some decision-making. After that, problems
will have fewer hints as you are expected to apply the skills learned in the earlier
problems.

\begin{quote}
  \color{blue}
  To assist in your learning, you should have a notebook where you write out your solution
  to each problem, along with notes on what you have learned. You will find that you refer
  to it often.
\end{quote}

When you solve actual numbered problems, you will write functions with names like
\pycode|ex101| (short for ``exercise 101''. You can then run the function interactively,
as you did with \pycode|example1|, but you can also do something even better: run it
against test data and judging data. These ideas were explained in the Introduction.

% ---------------------------------------------------------------- 101 Classify a triangle

\clearpage

\questionheader{\theproblemnumber\ Classify a triangle \workedexample}

\Question Read three integers representing sides of a triangle and
determine whether the triangle is equilateral, isosceles, or scalene.

\Sample

\minipagesthree{\sample{0.25}{17\\14\\13}{0.25}{scalene}}
               {\sample{0.25}{13\\13\\8}{0.25}{isosceles}}
               {\sample{0.25}{5\\5\\5}{0.25}{equilateral}}

\Solution

\begin{pythoncode} 
  def ex101(IN, OUT):
    a = int(IN.readline())
    b = int(IN.readline())
    c = int(IN.readline())

    if a == b and a == c:
      answer = 'equilateral'
    elif a == b or a == c or b == c:
      answer = 'isosceles'
    else:
      answer = 'scalene'

    print(answer, file=OUT)
\end{pythoncode}

\Explanation\ Lines 2--4 read the three lines in \IN\ into integer variables $a$, $b$ and
$c$. Lines 6--11 determine the answer, and Line 13 writes the answer to \OUT.

This program is very much about making decisions (\pycode|if|/\pycode|elif|/\pycode|else|)
with a small amount of data. Note the use of \pycode|and| and \pycode|or| in making the
decisions. \pycode|elif| is short for ``else if''.

Line 1 wraps this code up in a function called \pycode|ex101|, which is what the
\textbf{learninformatics} environment expects to find when you...well, see below.

\Running

The Introduction explained the features of the \textbf{learninformatics} environment. Now
we see them for ourselves.

After you have \emph{Run} your code in replit.com, try the following in the console.
\begin{itemize}
  \item \pycode|l.run(101)|, then use the keyboard to enter the values \texttt{15},
    \texttt{15} and \texttt{13}. The answer \texttt{isosceles} should immediately be
    displayed.
  \item \pycode|l.run(101, "15\n19\n20\n")|. This puts the data in directly, instead of
    waiting for you to type it. The answer \texttt{scalene} should immediately be
    displayed.
  \item \pycode|l.samples(101)|. This runs your code against the three samples shown
    above and checks that the correct output is produced. If the output is incorrect, or
    if an error occurs, a helpful report is printed so you can try to work out what went
    wrong.
  \item \pycode|l.judge(101)|. This runs your code against a greater variety of inputs and
    checks for the correct output. The inputs remain secret, the information printed if
    something goes wrong is quite basic. If your code passes all tests, a success token is
    printed as a reward.
\end{itemize}

Note that the name of the function containing your solution is important. If you call it
\pycode|training101| or \pycode|unicorn|, or anything other than \pycode|ex101|, then the
actions above will not work.

\begin{tcolorbox}
  As you proceed through the exercises, you will have many \texttt{def}s in your code, one
  for each exercise. This is excellent: a record of your work that you can look back on.
\end{tcolorbox}


% -------------------------------------------------------------- 102 Calculate a gradient

\clearpage

\questionheader{\theproblemnumber\ Calculate a gradient \workedexample}

\Question Read four floats representing points $(x_1,y_1)$ and
$(x_2,y_2)$ and find the \emph{gradient} of the interval $AB$, which is typically called
$m$ and is calculated as follows:
\[ m = \frac{\mathrm{rise}}{\mathrm{run}} = \frac{y_2 - y_1}{x_2 - x_1}.\]

Output the gradient, rounded to three decimal places.  If the gradient is undefined
(because the run is zero), output \texttt{undefined}.

\Sample

\minipagesthree{\sample{0.25}{5\\1\\10\\3.5}{0.25}{2.500}}
               {\sample{0.25}{-2.76\\-1.01\\3.14159\\-10.559}{0.25}{-1.693}}
               {\sample{0.25}{3.7\\9.5\\3.7\\17}{0.25}{undefined}}

\Explanation\ The answers given reflect the following calculations:

\vspace{-5mm}
\minipagesthree{\[\frac{3.5-1}{10-5} = 2.5\]}
               {\[\frac{-10.559 - (-1.01)}{3.14159 - (-2.76)} = -1.692594\ldots\]}
               {\[\frac{17 - 9.5}{3.7 - 3.7} = \frac{7.5}{0}\]}


\Solution

\begin{pythoncode} 
  def ex102(IN, OUT):
    x1 = float(IN.readline())
    y1 = float(IN.readline())
    x2 = float(IN.readline())
    y2 = float(IN.readline())

    rise = y2 - y1
    run  = x2 - x1

    if run == 0.0:
      answer = 'undefined'
    else:
      answer = rise / run

    print(round(answer,3), file=OUT)
\end{pythoncode}

\Explanation\ Lines 2--5 read the four floats from \IN\ and assign them to sensible
variable names. Lines 7--8 calculate the rise and run, which are important values in
determining the gradient. Lines 10--13 determine the answer, looking out for the undefined
case. Line 15 outputs the answer to \OUT, rounding it to three decimal places as required.

\Running\ The function is named \pycode|ex102|, so you can:
\begin{itemize}
  \item Run the code interactively with \pycode|l.run(102)|.
  \item Test the correctness of the code against some data with \pycode|l.samples(102)|
    and \pycode|l.judge(102)|.
\end{itemize}
