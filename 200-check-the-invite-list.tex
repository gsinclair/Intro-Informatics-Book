
\questionheader{Check the invite list \workedexample}

\Question\ Iliana is planning a party. Like, the best party e-vah! She already has a draft
list of who she wants to invite, but now her friend Alona mentions a new name. Or is it
new!? We'll have to check the invite list to find out.

\Input\ From \IN\ you will read:
\begin{itemize}
  \item a name $Q$, the person whose invite-status is being questioned;
  \item a positive integer $N$, the number of people on the draft invite list;
  \item $N$ names $I_1$ to $I_N$, representing people who are on the draft invite list.
\end{itemize}

\Output\ A single line to say whether the name $Q$ appears on the draft invite list, and
if so, where. The samples make this clear.

\Sample

{\small
\minipagestwo{%
  \sample{0.2}{Kevin\\5\\Jenny\\Tonya\\Sandy\\Erin\\Mike}
         {0.6}{Kevin is not yet invited}
}{%
  \sample{0.2}{Kevin\\6\\Jenny\\Tonya\\Sandy\\Kevin\\Erin\\Mike}
         {0.6}{Kevin is \#4 on the list}
}
}

\Explanation\ In the first sample, we are determining whether Kevin appears in the list
Jenny, Tonya, Sandy, Erin and Mike. Kevin is \emph{not} among those names, hence the
output \texttt{Kevin is not yet invited}. In the second sample, Kevin \emph{is} among the
names and appears in position number 4, hence the output \texttt{Kevin is \#4 on the
list}.

\Scratch\ This exercise brings you two new skills:
\begin{itemize}
  \item reading and working with strings; and
  \item keeping track of where we are in the list.
\end{itemize}

Remember also that you can \pycode|break| out of a loop. In the second sample, once you
find Kevin in position 4 you know what output is required, so there's no need to continue
reading names.

\textbf{Reading strings} from \IN\ is very straightforward. Here's the summary:
\begin{pythoncode}
  name = IN.readline().strip()
\end{pythoncode}

There are two things to notice about this use of \pycode|readline()| compared to what
we usually do.
\begin{enumerate}
  \item There is no \pycode|int(...)| or \pycode|float(...)| because we are \emph{not}
    dealing with numerical data.
  \item We append \pycode|.strip()| to remove the newline character at the end.
\end{enumerate}

To explain the second item further, \pycode|readline()| reads a \emph{line} of input,
exactly as it says. And a line is terminated by a \emph{newline} character, which is what
gets collected when you hit the Enter or Return key. So you might like to think that
reading the second line of the sample input(s) gives you \pycode|'Kevin'|, but
unfortunately it does not; you get \pycode|'Kevin\n'|. Calling \pycode|.strip()| on this
string removes any whitespace from the left or right, giving us the \pycode|'Kevin'| that
we want.

Here is a simple example you can try. Run it with \pycode|l.run(newline_example)| and type
two strings as input (e.g. \texttt{Hello} and \texttt{Goodbye}).
\begin{pythoncode}
  def newline_example(IN, OUT):
    a = IN.readline()
    b = IN.readline().strip()
    print('a: {repr(a)}  b: {repr(b)}', file=OUT)
\end{pythoncode}

The special function \pycode|repr|%
\footnote{See docs.python.org/3/library/functions.html}
gives you a programmer's view of what an object is, thus it explicitly \emph{shows} you
the newline character instead of \emph{printing} it.

\textbf{Working with strings} for this problem means asking whether two strings are the
same. This is no different from asking whether two integers are the same. You can try the
following in the console:
\begin{pythoncode}
  'Kevin' == 'Kevin'        # True
  'Kevin' == 'Andy'         # False
  'Kevin' > 'Andy'          # True (alphabetical order)
\end{pythoncode}

\textbf{Keeping track of where we are} is also easy enough, but to ensure understanding we
need to look at a few ways of doing it. To develop this, we assume that our code will look
something like this
\begin{pythoncode}
  for i in range(N):
    name = IN.readline().strip()
    # ...
\end{pythoncode}
and build around it.

Before we look at any code, consider the trace table we are trying to achieve, using the
second sample as input.

\begin{inlinetable}
  \begin{tabular}{cl}
    \toprule
    position & name  \\
    \midrule
    1        & Jenny \\
    2        & Tonya \\
    3        & Sandy \\
    4        & Kevin \\
    \bottomrule
  \end{tabular}
\end{inlinetable}
\medskip

The fifth and sixth names are not shown as we would stop reading when we find
Kevin, and we would have the information we need to report that he is \#4 in the list.

The first approach is to have a variable called \pycode|position| that we initialise to 1
and update each time we go through the loop.
\begin{pythoncode}
  position = 1
  for i in range(N):
    name = IN.readline().strip()
    # ...
    position = position + 1
\end{pythoncode}

This could be written with a \pycode|while| loop instead of a \pycode|for| loop, but it's
not an improvement.

\begin{pythoncode}
  position = 1
  i = 0
  while i < N:
    name = IN.readline().strip()
    # ...
    position = position + 1
    i = i + 1
\end{pythoncode}

However, it occurs to us that we might not need \emph{both} \pycode|position| and
\pycode|i|. The purpose of \pycode|i| is to run the loop the correct number of times; this
duty could be performed by \pycode|position| instead.

\begin{pythoncode}
  position = 1
  while position <= N:
    name = IN.readline().strip()
    # ...
    position = position + 1
\end{pythoncode}

That's better. And now that we have a conceptually clean approach, we can translate it
back into a \pycode|for| loop.

\begin{pythoncode}
  for position in range(1, N+1):
    name = IN.readline().strip()
    # ...
\end{pythoncode}

In that final piece of code, we don't need to assign or update \pycode|position| manually;
the \pycode|for| loop takes care of it for us. We just need to specify what we want the
range to be.

\textbf{We solve the problem} by the following steps.
\begin{enumerate}
  \item Read $N$ and $Q$ and set them to variables \pycode|N| and \pycode|target_name|.
  \item Set a variable \pycode|found| to \pycode|False|. We need to know whether we found
    the name or not.
  \item Use \pycode|for position in range(1, N+1):| to loop $N$ times while keeping track
    of the position $(1,2,3,\dots,N)$.
  \item Each time through the loop, we read a name (and strip the newline). If it's equal
    to \pycode|target_name|, set \pycode|found| to \pycode|True| and break out of the
    loop. If it's not equal to \pycode|target_name|, there's nothing special we need to
    do, just let the loop run again.
  \item When the loop finishes (either by \pycode|break| or by running out of items to
    read), we need to print the answer, which depends on whether the target name was
    \pycode|found| or not.
\end{enumerate}

\Solution

\begin{pythoncode}
  target_name = IN.readline().strip()
  N = int(IN.readline())
  found = False

  for position in range(1, N+1):
    name = IN.readline().strip()
    if name == target_name:
      found = True
      break

  if found:
    print(f'{target_name} is #{position} on the list', file=OUT)
  else:
    print(f'{target_name} is not yet invited', file=OUT)
\end{pythoncode}

\bigskip
Here are trace tables for the two examples.
\medskip

\minipagestwo{%
  \begin{center}
    \begin{tabular}{cll}
      \toprule
      position & name  & found \\
      \midrule
      1        & Jenny & False \\
      2        & Tonya & False \\
      3        & Sandy & False \\
      4        & Erin  & False \\
      5        & Mike  & False \\
      \bottomrule
    \end{tabular}
  \end{center}
}{%
  \begin{center}
    \begin{tabular}{cll}
      \toprule
      position & name  & found \\
      \midrule
      1        & Jenny & False \\
      2        & Tonya & False \\
      3        & Sandy & False \\
      4        & Kevin & True  \\
      \bottomrule
    \end{tabular}
  \end{center}
}
