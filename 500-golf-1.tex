
\questionheader{\theproblemnumber\ Golf (1)}

\Question\ Two people play a short game (nine holes) of golf, and you must write a program
to determine the winner.

Normally, the winner is the person with the lowest total score, where ``score'' means
``number of shots required to complete all holes''. But sometimes, players prefer to count
\emph{holes} instead of \emph{shots} to determine the winner. And this is one of those
times.

Say Jennifer won five holes, Kate won three holes, and one hole was tied. Then the result
would be \texttt{Jennifer won by 2 holes}. If Jennifer and Kate both won four holes and
one was tied, then we have to decide the winner by overall points, and the result might be
\texttt{Kate won by 3 points}. If both holes and points are tied, then the result is
\texttt{Jennifer and Kate tied}.

\Input\ From \IN\ you will read data from \emph{three lines}:
\begin{enumerate}
  \item The names of the two players
  \item The scores for the first player (nine integers)
  \item The scores for the second player (nine integers)
\end{enumerate}

\Output\ To \OUT\ you will write the result, as suggested by the paragraph above.

\Sample

\sample{0.3}{Jennifer Kate\\5 4 5 3 3 4 6 4 5\\4 4 5 4 4 4 5 4 4}
       {0.3}{Kate won by 1 hole}

\vspace{12pt}
\sample{0.3}{Jennifer Kate\\5 4 5 3 3 4 6 4 4\\4 4 5 4 4 4 5 4 4}
       {0.3}{Jennifer won by 2 points}

\vspace{12pt}
\sample{0.3}{Jennifer Kate\\5 4 5 3 3 4 6 4 4\\4 4 5 4 4 4 5 4 4}
       {0.3}{There was no winner}

\note{TODO: massage the data!}

\Explanation

\begin{itemize}
  \item In the first sample, Jennifer won two holes and Kate won three holes, so a winner
    can be determined by holes alone.
    \item In the second sample, Jennifer and Kate won three holes each, so points total
      points need to be considered. The points for Jennifer and Kate were 38 and 40
      respectively \note{TODO: check!}, and a lower score is better.
    \item In the third sample, Jennifer and Kate are equal on both holes and points.
\end{itemize}

\Scratch\ Ensure you've read the notes at the start of this chapter about reading multiple
values from one line.

