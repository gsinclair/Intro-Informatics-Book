
\questionheader{\theproblemnumber\ Cute numbers}

\Question\ \grey{(Sourced from ORAC)}\quad For you, numbers have personalities. The number
4 is elegant, 18 is strong and 42 is enigmatic.  And, of course, any number ending in 0 is
cute.  The more zeroes at the end of a number, the cuter that number is. Therefore $70$,
$36\,640$ and $1\,800\,090$ are only a little bit cute (ending in just one zero), whereas
400 and $99\,200$ are very cute (ending in two zeroes) and $30\,000$ is really really cute
(ending in four zeros).

Your task is to read in a number $N$ and determine how many zeros are at the end of that
number, so you can tell just how cute the number is.

\Input\ The first line of input will consist of the single integer $d$, telling you how
many digits are in the number $N$. You are guaranteed that $1 \le d \le 100\,000$.

Following this will be $d$ additional lines, each containing a single digit (0, 1, 2, 3,
4, 5, 6, 7, 8 or 9). These will be the digits of $N$, written from left to right. You are
guaranteed that the first digit of $N$ will not be zero.

\Output\ You must write a single integer as output, representing the number of zeroes at
the end of $N$.

\Sample

\minipagestwo{%
  \sample{0.4}{5\\9\\9\\2\\0\\0}
         {0.4}{2} }{%
  \sample{0.4}{7\\1\\9\\0\\0\\0\\9\\0}
         {0.4}{1} }


\Explanation The first example describes $N = 99\,200$, which ends in two zeros. The second
example describes $N = 1\,800\,090$, which contains many zeroes within but has only one zero
at the end.

