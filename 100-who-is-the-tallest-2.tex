\questionheader{\theproblemnumber\ Who is the tallest (2)?}

\emph{This question takes another look at \futurereference{Problem 101}.}

\Question\ In the schoolyard you determine who is the tallest by standing back to back.
Among your online friends, though, the only way to find this out is to compare numbers.
There are four people in your group, and so when everyone has entered their height in
centimetres, all that remains is to pick the largest number out of the list.

Your program will read four numbers from \IN\ and write the largest number to \OUT.

\Sample

\sample{0.2}{147\\165\\171\\168}{0.2}{171}

\Scratch\ Just like the earlier problem (\futurereference{Tallest person (1)}), we need to
read four integers from \IN. But this time, instead of making a lot of comparisons between
$a$, $b$, $c$ and $d$, we will use Python's built-in \code{max} function.

If you try the following Python expressions in your shell...

\begin{pythoncode}
    max(5, 1, 2, 9)
    max(7, 2, 5, 11, 6)
    max(-51, 73, 33, -1, 0, 12)
\end{pythoncode}

...then you will learn with some relief that Python has pretty much solved the ``Who is
the tallest?'' problem for us.

\Solution\ Make a new function \emph{in the same file}:

\begin{pythoncode}
    def train102(IN, OUT):
        a = int(IN.readline())
        b = int(IN.readline())
        c = int(IN.readline())
        d = int(IN.readline())

        answer = max(a, b, c, d)

        print(answer, file=OUT)
\end{pythoncode}

Use the tag \problemtagtt\ when you test and judge your code.

\Afterword\ The \pycode|max()| function can take any number of arguments, or it can find
the maximum of a (potentially very long) list. These are details for later. But here are
two teasers that you can run in the shell and think about.

\begin{pythoncode} 
  max(n*n for n in [-6, -4, -1, 0, 3])
  max(len(x) for x in ["Emma", "Caitlin", "Tim"])
\end{pythoncode}

