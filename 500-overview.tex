
\chapter{Problems 5xx: two-dimensional data}

Intro text...

% ---------------------------------------------------------------- Cash grab (1)

\clearpage

\questionheader{\theproblemnumber\ Cash grab (1)}

Behind the gates of the walled city \emph{Adventis} lies an enormous $8 \times 10$ board
with 80 tiles, each one of them filled with coins. You are invited to choose which tile
to stand on and will be given one minute to scoop up as much cash as you can. Taking a
look from on high, you reason that wherever you stand, one minute will only afford you
enough time to get the cash on that tile and on all the immediately surrounding tiles.
That is, nine tiles in total.

On a scrap of paper, you jot down your estimate of how much cash is on each tile, and you
have only moments to decide where you will be placed.

\Input Eight rows of ten non-negative integers, each representing the cash amount on one
tile. The rows are numbered 1--8 and the columns 1--10, but these do not appear in the input.

For simplicity's sake, as this is your first exposure to two-dimensional data, all border
tiles will have a value of \texttt{0}.

\Output Two numbers \texttt{row col} identifying the most profitable place to stand.

\Sample

\sample{0.5}{0 0 0 0 0 0 0 0 0 0\\
             0 1 2 1 3 2 4 3 1 0\\
             0 4 1 2 1 3 0 2 0 0\\
             0 0 5 0 3 2 1 3 1 0\\
             0 2 7 8 5 4 2 0 3 0\\
             0 1 6 3 4 2 0 0 5 0\\
             0 3 0 2 2 1 2 2 3 0\\
             0 0 0 0 0 0 0 0 0 0}
       {0.2}{5 4}

\Explanation\ The location $(5,4)$ (in row-column notation) is the square with value
\texttt{8}, and the nine cells within reach are \texttt{5 0 3 7 8 5 6 3 4}, which clearly
produces a higher total than any other starting cell.%
\footnote{We can see the answer by eye in this case, but with different data it would not
be so easy.}

\Scratch\ We need to simulate standing in many different places and counting the
surrounding cash, and keep a running ``best location''. How many places do we need to try?
Well, the zeros on the outside mean there's no point standing on them. There's no point
standing \emph{next to} a row or column of zeros either, so that means we try all
locations in the rectangle $(2,2)$--$(8,8)$.

But first, we need to load the data in. We will create a list called \texttt{DATA} which
is really a list of \emph{rows}, each of which is a list of values. That is, when we're
done, we'll have (referring to the sample data):

\begin{pythoncode}
  DATA == [[0,0,0,0,0,0,0,0,0,0],    # Row 1 (index 0)
           [0,1,2,1,3,2,4,3,1,0],    # Row 2 (index 1)
           [0,4,1,2,1,3,0,2,0,0],    # Row 3 (index 2)
           # ...
           [0,0,0,0,0,0,0,0,0,0]]    # Row 10 (index 9)
\end{pythoncode}

Thus we have two-dimensional data as a list of lists. If we want to access the \texttt{4}
at row 2, column 7, then we use \pycode|DATA[1][6]|, noting zero-based indexing. Just
typing \pycode|DATA[2]| gives us the entirety of row 3.

Here is a way to get the values from \IN\ into our variable \pycode|DATA|.

\begin{pythoncode}
  DATA = []                # Start with an empty list and append each row.
  for i in range(8):
    row = [int(x) for x in IN.readline().split()]
    DATA.append(row)
\end{pythoncode}

Now to actually access all the data we want, from $(2,2)$ down to $(6,8)$, we can use a
loop within a loop. This is the normal way to deal with two-dimensional data.

\begin{pythoncode}
  for row in range(2,7):
    for col in range(2,9):
      print(f'Row {row+1}, Column {col+1}: {DATA[row][col]}')
\end{pythoncode}

It is worth trying this code to ensure you know what it does and why. Note that we are
translating between the 0-based indexing of lists and the 1-based language of the problem.

The double-loop above simulates standing in the 49 different locations from $(2,2)$ to
$(6,8)$. So what do we want to do in each location? We want to count all the cash in that
location and its eight neighbours.

\begin{pythoncode}
  for row in range(2,7):
    for col in range(2,9):
      cash = DATA[row-1][col-1] + DATA[row-1][col] + DATA[row-1][col+1] + \
             DATA[row]  [col-1] + DATA[row]  [col] + DATA[row]  [col+1] + \
             DATA[row+1][col-1] + DATA[row+1][col] + DATA[row+1][col+1]
\end{pythoncode}

With code like that, we might prefer to use \texttt{r} and \texttt{c} instead of
\texttt{row} and \texttt{col}. That will appear in the solution.

More than just count the cash, though, we need to see whether this is the best cash amount
we've encountered so far, and if so, remember where it is. For that we will use two
variables: \texttt{bestcash} and \texttt{location}. We will store the location as a
(row,column) pair.%
\footnote{Python conveniently allows us to store pairs or triples or ... of values in
round brackets. These are called \emph{tuples}, and once you've created them you can't
change them. You'll learn why not when you encounter the \emph{dictionary} data type.}

\Solution

\begin{pythoncode}
  def train501(IN, OUT):
    DATA = []
    for i in range(8):
      row = [int(x) for x in IN.readline().split()]
      DATA.append(row)
    
    bestcash = 0
    location = (0,0)

    for r in range(2,7):
      for c in range(2,9):
        cash = DATA[r-1][c-1] + DATA[r-1][c] + DATA[r-1][c+1] + \
               DATA[r]  [c-1] + DATA[r]  [c] + DATA[r]  [c+1] + \
               DATA[r+1][c-1] + DATA[r+1][c] + DATA[r+1][c+1]
        if cash > bestcash:
          # We have a new potential answer.
          bestcash = cash
          location = (r+1, c+1)

    row, column = location
    print(row, column, file=OUT)
\end{pythoncode}

\Explanation\ Much of the code was explained in \emph{Scratch} above. In Lines 7--8 we
initialise the two variables that help us find the best location, and in Lines 15--18 we
use them. In Line 18 we add 1 to the row and column because that is the numbering system
used in the problem. In Line 20 we \emph{unpack the tuple} into its two individual values
so that we can print them out with a space between. If we had called
\pycode|print(location)| then the output would be \texttt{(5,4)}.


% ---------------------------------------------------------------- Cash grab (2)

\clearpage

\questionheader{\theproblemnumber\ Cash grab (2)}

The question is the same as \emph{Cash grab (1)} but the board size is now variable. The
first line of \IN\ will contain the values $R$ and $C$, the number of rows and columns,
respectively. Following this will be $R$ lines, each containing $C$ non-negative integers.

\Sample

\sample{0.3}{...}
       {0.6}{...}

\Solution

\begin{pythoncode}
  def train501(IN, OUT):
    pass
\end{pythoncode}


% ---------------------------------------------------------------- Cash grab (3)

\clearpage

\questionheader{\theproblemnumber\ Cash grab (3)}

Variable size.

\Sample

\sample{0.3}{...}
       {0.6}{...}

\Solution

\begin{pythoncode}
  def train501(IN, OUT):
    pass
\end{pythoncode}

