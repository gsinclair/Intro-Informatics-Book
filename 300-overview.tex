
\chapter{Looping to solve problems}

In Chapter \futurereference{2, or whatever}, you learned to code \emph{loops} for the
purpose of reading an arbitrary number of lines of data. But the problems you solved were
still quite direct in nature. For example, you might read many lines like \texttt{circle}
followed by \texttt{40}, but the act of calculating $\pi(40)^2$ is still very
straightforward.

In this chapter we will generally read a small fixed amount of data but will need to use
loops to solve the problem itself. This is something that beginners tend to find very
difficult. So let's break it down with two simple examples that will be developed fully in
the first two problems:
\begin{itemize}
  \item How many factors does a number (42, say) have? The simplest way to answer this
    question is to consider every possible number between 1 and 42 (inclusive) and count
    how many of them divide 42 evenly, which in computing terms means leaving a remainder
    of zero. Thankfully, Python makes it easy to operate on the number 1, then the number
    2, then 3, and so on until we reach 42.
  \item How far does Sarah%
    \footnote{See \futurereference{Jogging (1)} if you haven't already}
    run in (say) 100 days, given she runs \SI{500}{\m} on the first day and
    increases her run by \SI{75}{\m} each day thereafter until she reaches \SI{1500}{\m}
    per day? Clearly, her running pattern over time will be \[
      500, 575, 650, \ldots, 1325, 1400, 1475, 1500, 1500, 1500, \ldots
    \] and we need to add 100 terms of that sequence. So we are doing \emph{something} 100
    times (adding a number to a running total, which begins at zero), but the details (how
    much we add) potentially changes each time.
\end{itemize}

Before we get stuck in to those two problems, we see below \textbf{two ways} of looping to
do something really simple: list the 12 times tables up to 5.

\begin{pythoncode}
  def times_tables_example_1():
    n = 1
    while n < 6:
      print(f'12 x {n} = {12*n}')
      n = n + 1
    print('Done')
\end{pythoncode}

If you type this in and run your code, you can then run \pycode|times_tables_example_1()|
in the shell and see the output:

\begin{quote}
  \begin{verbatim}
    12 x 1 = 12
    12 x 2 = 24
    12 x 3 = 36
    12 x 4 = 48
    12 x 5 = 60
    Done
  \end{verbatim}
\end{quote}

In the code above, $n$ is a \emph{loop counter} that takes on the values 1, 2, 3, 4, 5. It
takes on these values because we explicitly set it to 1, then the loop runs so long as $n
< 6$, and we explicitly increment it each time the loop runs. Thus Lines 4--5 (the
\emph{loop body}) run five times, controlled by the \emph{loop condition} on Line 3. When
the loop condition decides enough is enough (that is, the condition $n < 6$ becomes
\pycode|False|), control passes to Line 6, and the code then finishes.

The description above makes two mentions of the variable $n$ being manipulated
\emph{explicitly}. Python offers an \emph{implicit} alternative, where we basically say
``I want $n$ to range from 1 to 5'' and the low-level details are taken care of for us.
This is demonstrated in the code below.

\begin{pythoncode}
  def times_tables_example_2():
    for n in range(1,6):
      print(f'12 x {n} = {12*n}')
    print('Done')
\end{pythoncode}

The output from our two code listings is exactly the same. Beginners often prefer the
first style because you can see everything that is happening. It doesn't take long until
they prefer the second style, though, because looping is such a common occurence in
programs that we tend to appreciate it being supported directly.

Note that \pycode|range(1,6)| produces the values $1,2,3,4,5$ as we wish. It's known as an
inclusive-exclusive range: the first value (1) is \emph{included}; the second value (6) is
\emph{excluded}. There are good reasons for this that won't be detailed here.%
\footnote{Consider that, in a sense, \pycode|range(10,25) + range(25,30) == range(10,30)|}.

Here are a few more code snippets that you can run if you like to get a bit more
experience with how \pycode|for| loops work.

\begin{pythoncode}
  def for_loop_examples():
    for x in range(-8,9):
      print(f'{x} squared is {x*x}')

    for name in ['Steven', 'Peta', 'Jessica']:
      print(f'Hi {name}!!!')

    for m in range(5):
      print(f"This is printed five times. I don't about m, but m = {m}")

    import time
    for t in range(4,20,3):
      print(f'Sleeping for {t*100} milliseconds')
      time.sleep(t * 100)
\end{pythoncode}

As shown above, \pycode|range| can take either one, two or three arguments. To be clear:

\begin{center}
  \begin{tabular}{ll}
    Call this... & ...and you'll get...\\
    \pycode|range(5)|       & $0,1,2,3,4$ \\
    \pycode|range(3,9)|     & $3,4,5,6,7,8$ \\
    \pycode|range(4,14,2)|  & $4,6,8,10,12$ \\
    \pycode|range(10,0,-1)| & $10,9,8,7,6,5,4,3,2,1$ \\
  \end{tabular}
\end{center}

Just for fun, we can finish this introduction with a blast off, implemented two ways.

\begin{pythoncode}
  def blastoff_1():
    import time
    for n in range(10,0,-1):
      print(n)
      time.sleep(1000)
    print("Blast off!!!")

  def blastoff_2():
    import time
    n = 10
    while n > 0:
      print(n)
      time.sleep(1000)
      n = n - 1
    print("Blast off!!!")
\end{pythoncode}


% ---------------------------------------------------------------- Cash grab (1)

\clearpage

\questionheader{\theproblemnumber\ How many factors? \workedexample}

\Question\ Given a positive number $N$, determine how many factors it has.

\Input\ A single integer $N>0$.

\Output\ A single integer representing the number of factors of $N$.

\Sample

\minipagesthree{%
  \sample{0.3}{12}{0.3}{6}
}{%
  \sample{0.3}{441}{0.3}{9}
}{%
  \sample{0.3}{73}{0.3}{2}
}

\Explanation\ The number 12 has six factors: $1,2,3,4,6,12$. The number 441 has 9 factors:
$1,3,7,9,21,49,63,,147,441$. The prime number 73 has two factors: $1,73$.

\Scratch\ To get the factors of 12, we will consider every number from 1 to 12.%
\footnote{If we were smarter, we'd consider 1 to 6, then count 12 as an automatic factor.
  And there's an even smarter way than that!  But we're going for simplicity here.}
Here is a logical setting out of what needs to be done. First of all, initialise
\emph{count} to zero. Then let $f$ (called $f$ because it's a potential factor) take on
values 1 to 12 as the table shows.

\begin{center}
  \begin{tabular}{clc}
    \toprule
    \bi{f} & \textbf{factor of 12?} & \bi{count} \\
    \midrule
    1  & \pycode|True|  & 1  \\
    2  & \pycode|True|  & 2  \\
    3  & \pycode|True|  & 3  \\
    4  & \pycode|True|  & 4  \\
    5  & \pycode|False| &    \\
    6  & \pycode|True|  & 5  \\
    7  & \pycode|False| &    \\
    8  & \pycode|False| &    \\
    9  & \pycode|False| &    \\
    10 & \pycode|False| &    \\
    11 & \pycode|False| &    \\
    12 & \pycode|True|  & 6  \\
    \bottomrule
  \end{tabular}
\end{center}

When the loop has completed, we have $count = 6$ and that is the answer we write to \OUT.

Here is a pseudo-code%
\footnote{Pseudo-code is simply code \emph{ideas} expressed in general without using a
  specific programming language.}
algorithm for what we need to do.

\begin{quote}
  \begin{verbatim}
    set Count to 0
    foreach f in 1..12 {
      if f is a factor of 12 {
        increase Count
      }
    }
    output Count
  \end{verbatim}
\end{quote}

To solve the problem, we need to do two things:
\begin{itemize}
  \item generalise from 12 to $N$, the number that we get from \IN;
  \item express \texttt{f is a factor of N} in Python.
\end{itemize}

The first is easy. The second requires familiarity with the \emph{modulus} operator, which
in Python (in fact, in most programming languages) is spelled \texttt{\%}. It gives you
the remainder from a division. Consider these examples:

\begin{quote}
  \begin{verbatim}
    10 % 4    --> 2       (10 divided by 4 leaves remainder 2)
    11 % 4    --> 3       (11 divided by 4 leaves remainder 3)
    12 % 4    --> 0       (12 divided by 4 leaves no remainder -- it's divisible)
  \end{verbatim}
\end{quote}

Thus \texttt{f is a factor of N} translates into Python as \pycode|N % f == 0|.

\Solution

Here is the algorithm expressed in Python, in both looping styles.

% NOTE - Ensure correct function names

\begin{minipage}{\textwidth}
  \begin{pythoncode}
    def train301(IN, OUT):
      N = int(IN.readline())
      count = 0
      for f in range(1,N+1):
        if N % f == 0:
          count = count + 1
      print(count, file=OUT)

    def train301(IN, OUT):
      N = int(IN.readline())
      count = 0
      f = 1
      while f <= N:
        if N % f == 0:
          count = count + 1
        f = f + 1
      print(count, file=OUT)
  \end{pythoncode}
\end{minipage}
