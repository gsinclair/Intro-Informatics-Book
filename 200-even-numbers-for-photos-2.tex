
\questionheader{Even numbers for photos! (2)}

\Question\ You tried, but the difficulties of arranging photos for the wedding is just all
too hard, even with the help of your program. There are just too many scattered chains of
odd groups. So you decide to write a new program to focus on just the biggest problem.
Looking at the same example data at the previous problem \[
  \mathbf{7}
  \quad 8 \quad 10 \quad 8
  \quad \mathbf{11} \quad \mathbf{13} \quad \mathbf{11} \quad \mathbf{19}
  \quad 4
  \quad \mathbf{9} \quad \mathbf{15}
  \quad 6 \quad 8
\]
we see that the longest chain $[11,13,11,19]$ has length 4 and begins in position 5. This
information (length and starting position of the longest chain) is what you want to
uncover this time, because it will tell you the size of your greatest problem and where to
find it.

\Input\ The input for this exercise follows the same specification as the input for the
previous exercise.

\Output\ The length and starting location of the longest chain, formatted as shown in the
samples. If there are no chains of odd groups, both values will be zero.

\Sample

{\small
\minipagesthree{
  \sample{0.25}{13\\7\\8\\10\\8\\11\\13\\11\\19\\4\\9\\15\\6\\8}
         {0.4}{Length:\ 4\\Starts:\ 5}
}{
  \sample{0.25}{6\\5\\3\\4\\8\\6\\3}
         {0.4}{Length:\ 2\\Starts:\ 1}
}{
  \sample{0.25}{7\\4\\6\\4\\2\\6\\4\\6}
         {0.4}{Length:\ 0\\Starts:\ 0}
}
} % end small

\Explanation\ The first sample represents the example given in the question. The second
sample has its longest odd chain $[5,3]$ with length 2 starting in position 1. The third
sample has no odd chains, so it has the output specified earlier.

\Scratch\ We have seen before (\futurereference{Check the invite list}) how to track our
position in the list of data, so we can use that approach again:
\begin{pythoncode}
  for position in range(1, N+1):
    size = int(IN.readline())
    # ...
\end{pythoncode}

The values we need to track this time are:
\begin{itemize}
  \item a flag for whether we are in an odd chain or not (same as last time);
  \item the length of the current chain;
  \item the starting location of the current chain;
  \item the length of the longest chain seen so far; and
  \item the starting location of the longest chain seen so far.
\end{itemize}

Suggested variables then, in addition to \pycode|flag|, are \pycode|curr_len| (playing the
same role as \pycode|ngroups| did in the previous question), \pycode|curr_loc|
\pycode|longest_len| and \pycode|longest_loc|.

You might try to construct and complete a trace table before continuing. A completed trace
table is shown on the next page.


\clearpage % ------------------------------------------------------------------ new page
\questionheadercontinued{Even numbers for photos! (2)}

Here is the completed trace table.

\begin{inlinetable}
  \sisetup{detect-weight=true,detect-inline-weight=math}
  \begin{tabular}{S[table-format=2.0] S[table-format=2.0] c S[table-format=2.0]
    S[table-format=2.0] S[table-format=2.0] S[table-format=2.0]}
    \toprule
    {position} & {size}       & {f{}lag} & {curr\_len} & {curr\_loc} & {longest\_len} & {longest\_loc} \\
    \midrule                             
               &              & False    & 0           & 0           & 0              & 0              \\
    1          & \bfseries 7  & True     & 1           & 1           & 1              & 1              \\
    2          & 8            & False    & 0           & 0           & 1              & 1              \\
    3          & 10           & False    & 0           & 0           & 1              & 1              \\
    4          & 8            & False    & 0           & 0           & 1              & 1              \\
    5          & \bfseries 11 & True     & 1           & 5           & 1              & 1              \\
    6          & \bfseries 13 & True     & 2           & 5           & 2              & 5              \\
    7          & \bfseries 11 & True     & 3           & 5           & 3              & 5              \\
    8          & \bfseries 19 & True     & 4           & 5           & 4              & 5              \\
    9          & 4            & False    & 0           & 0           & 4              & 5              \\
    10         & 9            & False    & 1           & 10          & 4              & 5              \\
    11         & \bfseries 15 & True     & 2           & 10          & 4              & 5              \\
    12         & 6            & False    & 0           & 0           & 4              & 5              \\
    13         & 8            & False    & 0           & 0           & 4              & 5              \\
    \bottomrule
  \end{tabular}
\end{inlinetable}

As an exercise, you might like to complete another trace table for this problem. Here is
some sample data, where $N=17$.\[
  14 \quad 8 \quad 
  \mathbf{9} \quad \mathbf{15} \quad
  10 \quad 22 \quad 
  \mathbf{5} \quad \mathbf{21} \quad \mathbf{17} \quad 
  8 \quad
  \mathbf{27} \quad
  10 \quad
  \mathbf{3} \quad \mathbf{9} \quad \mathbf{11} \quad \mathbf{7} \quad \mathbf{17}
\]

The answer is on the next page.


\clearpage % ------------------------------------------------------------------ new page
\questionheadercontinued{Even numbers for photos! (2)}

Here is the completed trace table for the longer sample data.

\begin{inlinetable}
  \sisetup{detect-weight=true,detect-inline-weight=math}
  \begin{tabular}{S[table-format=2.0] S[table-format=2.0] c S[table-format=2.0]
    S[table-format=2.0] S[table-format=2.0] S[table-format=2.0]}
    \toprule
    {position} & {size}       & {f{}lag} & {curr\_len} & {curr\_loc} & {longest\_len} & {longest\_loc} \\
    \midrule                             
               &              & False    & 0           & 0           & 0              & 0              \\
    1          & 14           & False    & 0           & 0           & 0              & 0              \\
    2          & 8            & False    & 0           & 0           & 0              & 0              \\
    3          & \bfseries 9  & True     & 1           & 3           & 1              & 3              \\
    4          & \bfseries 15 & True     & 2           & 3           & 2              & 3              \\
    5          & 10           & False    & 0           & 0           & 2              & 3              \\
    6          & 22           & False    & 0           & 0           & 2              & 3              \\
    7          & \bfseries 5  & True     & 1           & 7           & 2              & 3              \\
    8          & \bfseries 21 & True     & 2           & 7           & 2              & 3              \\
    9          & \bfseries 17 & True     & 3           & 7           & 3              & 7              \\
    10         & 8            & False    & 0           & 0           & 3              & 7              \\
    11         & \bfseries 27 & True     & 1           & 11          & 3              & 7              \\
    12         & 10           & False    & 0           & 0           & 3              & 7              \\
    13         & \bfseries 3  & True     & 1           & 13          & 3              & 7              \\
    14         & \bfseries 9  & True     & 2           & 13          & 3              & 7              \\
    15         & \bfseries 11 & True     & 3           & 13          & 3              & 7              \\
    16         & \bfseries 7  & True     & 4           & 13          & 4              & 13             \\
    17         & \bfseries 17 & True     & 5           & 13          & 5              & 13             \\
    \bottomrule
  \end{tabular}
\end{inlinetable}
