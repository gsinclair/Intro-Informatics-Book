
\questionheader{Golf (2)}

This question varies the earlier \emph{Golf (1)} with an arbitrary number of players and
holes.

\Question\ A group of $N$ people play a game of golf (with $Q$ holes), and you must write
a program to determine the winner. As before, whoever wins the most holes is the winner.
If there is a tie for most holes won, then it's whoever took the fewest overall shots.

\Input\ The first line of \IN\ contains the integers $N$ and $Q$, the number of players
and number of holes, respectively. The next line contains $N$ space-separated strings,
which are the first names of all the players. Then there are $N$ lines containing $Q$
integers each, which are the players' scores on the $Q$ holes.

\Output
\begin{itemize}
  \item If one player won more holes than any other player, you will output a line like
    \texttt{John won by 4 holes}.
  \item If holes are tied but one player scored a lower total than any other, you will
    output a line like \texttt{Samantha won by 3 points}.
  \item Otherwise, you will output \texttt{There was no winner}.
\end{itemize}

\Sample

\sample{0.3}{4 7\\
             Sam Steve Terri Kate\\
             5 3 5 6 7 5 5\\
             4 3 5 4 5 5 4\\
             4 4 4 4 4 5 3\\
             3 3 4 5 4 3 4}
       {0.3}{Kate won by 1 hole}

\vspace{12pt}
\sample{0.3}{...}
       {0.3}{...}

\vspace{12pt}
\sample{0.3}{...}
       {0.3}{...}

\Explanation

\begin{itemize}
  \item In the first sample, the winner of the seven holes were (using player numbers
    1--4): 4, -, -, -, -, 4, 3. Thus player 4 (Kate) won by one hole.
    \item In the second sample, ...
    \item In the third sample, ...
\end{itemize}

