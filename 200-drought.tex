
\questionheader{Drought}

\Question\ \grey{(Sourced from ORAC)}\quad Years of drought have hit rural Australia hard.
With catchment levels at an all time low, you decide to purchase a rainwater tank. Soon
the winter rains arrive, and the tank slowly begins to fill.

You begin to wonder just when your tank will be entirely full. A friend
in the weather bureau has kindly lent you rainfall predictions for the next few days.
Given these predictions, and the size of your rainwater tank, write a program to determine
how many days your tank takes to fill.

\Input\ The first two lines of the input file will contain the values $N$ and $C$, where
$N$ is the number of days the weather predictions last, and $C$ is the capacity of your
rainwater tank in litres.  You are guaranteed that $1 \le N \le 1000$, and that $C$ is a
positive integer no greater than the total amount of rain that falls over the $N$ days.

The remaining $N$ lines of input will describe the rainfall levels for each day in order.
Each line will contain a single integer between 0 and 1\,000\,000: the amount of rain (in
litres) that will fall over your rainwater tank that day.

\Output\ Your output should consist of a single integer: the number of days until your
rainwater tank fills.

\Sample

\minipagestwo{%
  \sample{0.4}{6\\10\\2\\3\\3\\2\\2\\4}
         {0.4}{4} }{%
  \sample{0.4}{6\\11\\2\\3\\3\\2\\2\\4}
         {0.4}{5} }


\Explanation\ In both examples, the total rainfall changes as follows:

\begin{inlinetable*}
  \begin{tabular}{c S[table-format=2.0]}
    \toprule
    {Day} & {Running total (L)} \\
    \midrule
    1   & 2                 \\
    2   & 5                 \\
    3   & 8                 \\
    4   & 10                \\
    5   & 12                \\
    6   & 16                \\
    \bottomrule
  \end{tabular}
\end{inlinetable*}

Hence a 10-litre water tank is full after 4 days and an 11-litre water tank is full after
5 days.

\Scratch\ If you intialise a variable to track the running total to zero at the beginning
and keep it updated as you read in the data, you should be able to solve this.

Good to know: you can \pycode|break| out of a \pycode|for| loop just like you can a
\pycode|while| loop.
