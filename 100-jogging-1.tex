
\questionheader{Jogging (1)}

\Question\ Sarah is about to take up jogging. She will start jogging $D$ metres per day
and increase her daily distance by $I$ metres each day. When she reaches her target of $T$
metres per day, she will stop the daily increase and just keep a consistent jogging
pattern.

How many days after she starts will she reach her target?

\Input

The input contains three lines, all of which are values in metres:
\begin{itemize}
  \item $D$, the distance at which she starts jogging;
  \item $I$, the amount by which she increases her jog each day;
  \item $T$, her target daily jog, after which she doesn't increase it any further.
\end{itemize}

All values are integers, and you are guaranteed that $D < D+I < T$, that is, she will have
to apply at least two increases before reaching her target.

\Output\ You will write a single integer to \OUT, that being the number of days after
starting that she reaches her target.

\Sample

\minipagestwo{%
  \sample{0.4}{100\\50\\300}
         {0.4}{4} }{%
  \sample{0.4}{700\\20\\750}
         {0.4}{3} }


\Explanation In the first sample, her running pattern over the days goes like this:
\begin{center}
  \begin{tabular}{cc}
    Day & Distance (m) \\
    1   & 100\\
    2   & 150\\
    3   & 200\\
    4   & 250\\
    5   & 300
  \end{tabular}
\end{center}

She reaches her target on day 5, which is four days after she starts jogging, so the
answer is \texttt{4}.

In the second sample, her running pattern \emph{would} proceed 700, 720, 740, 760, but
\SI{760}{\m} actually exceeds her target of \SI{750}{\m}, so she would in fact run
\SI{750}{\m} on the fourth and subsequent days. In any case, she reached/exceeded her
target on day 4, which is three days after she started, so the answer is \texttt{3}.

\Scratch\ It is intended that you use simple mathematics to work this out. As a very large
hint, consider the expression $(300 - 100)/50$ in the first sample. As another hint, be
aware of the difference between \pycode|/| and \pycode|//| in Python. Finally, although
it's not strictly necessary, you might benefit from searching for examples of
\pycode|math.floor| and \pycode|math.ceil|.

