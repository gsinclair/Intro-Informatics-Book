
\questionheader{Sales stars}

\Question\ The end of the quarter is nigh, and it's time for the best salespeople in
\textsf{Tancorp} to bask in adulation as they are crowned this quarter's \emph{Sales
Stars}. Can you quickly work out who they are?

\Input\

\begin{itemize}
  \item The first line contains two positive integers $N$ and $T$. $N$ is the number of
    lines of data that follow, and $T$ is the sales target above which a person will be
    considered a \emph{Sales Star}.
  \item Following that are $N$ lines of data, one for each salesperson. A line of data
    contains the person's name followed by some positive integers, each representing the
    number of widgets that person sold in a month. There will be between 1 and 3
    (inclusive) such numbers in a line, because this represents the last quarter's worth
    of data, but some salespeople started at the \textsf{Tancorp} only a month or two ago.
\end{itemize}

\Output\ The names of all salespeople whose total sales exceed $T$. The names must appear
in the same order that they appears in the input.

\Sample

\sample{0.35}{5 90\\Tom 33 28 17\\Jerry 39 41 22\\Maddie 22 38\\Sharon 11 35 36\\Hall 40 45 41}
       {0.2}{Jerry\\Hall}

\Explanation\ The salespeople's totals are shown below.

\begin{inlinetable}
  \begin{tabular}{lS[table-format=3.0]}
    \toprule
    Name   & Total \\
    \midrule
    Tom    & 78    \\
    Jerry  & 102   \\
    Maddie & 60    \\
    Sharon & 82    \\
    Hall   & 126   \\
    \bottomrule
  \end{tabular}
\end{inlinetable}

\medskip
Jerry and Hall are the two who exceed the target of 90, hence the answer.

\Scratch\ You'll need some help separating the names from the numbers here. This is a job
for Python's \emph{list slicing}. The following snippet demonstrates all you need for this
problem.

\begin{pythoncode}
  data = ['Tom', '33', '28', '17']
  name = data[0]
  rest = data[1:]
\end{pythoncode}

That \pycode|data[1:]| means: take a slice from the \pycode|data| list, starting at index
1 and continuing until the end. Here are some other slices you might try out of interest.

\begin{pythoncode}
  data = [10, 11, 12, 13, 14, 15, 16, 17, 18, 19, 20]
  data[3:7]
  data[1:8:2]
  data[::2]
  data[1::2]
  data[5:0:-1]
  data[::-1]
\end{pythoncode}

