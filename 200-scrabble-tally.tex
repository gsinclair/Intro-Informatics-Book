
\questionheader{Scrabble tally}

\Question\ Four friends---Albert, Betty, Charlie and Dotty---have played a lot of Scrabble
games over the years. The competition has been pretty close, but they've all kind of
forgotten who has won how many games. Luckily, when moving house recently, Dotty unearthed
a long list of the winners of all their games! By processing this list, you will be able
to give them a full tally.

\Input\ The first line of \IN\ contains $N$, the number of winners listed. Following this
are $N$ lines, each of which contains the name \texttt{Albert}, \texttt{Betty},
\texttt{Charlie} or \texttt{Dotty}, representing the winner of one game.

\Output\ \OUT\ contains four lines, reporting a tally as seen in the sample.

\Sample

\sample{0.25}{10\\Dotty\\Albert\\Albert\\Charlie\\Dotty\\Albert\\Dotty\\Dotty\\Dotty\\Charlie}
       {0.2}{Albert: 3\\Betty: 0\\Charlie: 2\\Dotty: 5}

\Scratch\ In this problem, you are reading \emph{strings} for the first time. ``String''
is the computer-science word for textual data, presumably because the characters are
``strung'' together. To read a string, we could simply call \pycode|IN.readline()|, with
no \pycode|int(...)| or \pycode|float(...)| wrapper because we certainly do not want
Python to attempt to convert our names to numbers.

There's a sting quite literally in the tail, though, because when we read a line of data
from \IN, the resulting string has a newline character at the end, and we don't really
want that.
