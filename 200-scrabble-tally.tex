
\questionheader{Scrabble tally \stronghint}

\Question\ Four friends---Albert, Betty, Charlie and Dotty---have played a lot of Scrabble
games over the years. The competition has been pretty close, but they've all kind of
forgotten who has won how many games. Luckily, when moving house recently, Dotty unearthed
a long list of the winners of all their games! By processing this list, you will be able
to give them a full tally.

\Input\ The first line of \IN\ contains $N$, the number of winners listed. Following this
are $N$ lines, each of which contains the name \texttt{Albert}, \texttt{Betty},
\texttt{Charlie} or \texttt{Dotty}, representing the winner of one game.

\Output\ \OUT\ contains four lines, reporting a tally as seen in the sample.

\Sample

\sample{0.25}{10\\Dotty\\Albert\\Albert\\Charlie\\Dotty\\Albert\\Dotty\\Dotty\\Dotty\\Charlie}
       {0.2}{Albert:~~3\\Betty:~~~0\\Charlie:~2\\Dotty:~~~5}

\Scratch\ You know from the previous problem how to read strings (and remove the trailing
newline) and compare them. This problem is simply a matter of counting how many occurences
of four different strings there are.

If there are $N$ names in the input data then we need to call \pycode|readline()| $N+1$
times. Each time we read a line in the loop, we are reading a \emph{name}. So our code is
structured like this.

\begin{pythoncode}
  N = int(IN.readline())

  for i in range(N):
    name = IN.readline().strip()
\end{pythoncode}

But what about the counting? We are counting occurences of Albert, Betty, Charlie and
Dotty, so let's have variables named \pycode|a|, \pycode|b|, \pycode|c| and \pycode|d|.

\begin{pythoncode}
  N = int(IN.readline())
  a = 0
  b = 0
  c = 0
  d = 0

  for i in range(N):
    name = IN.readline().strip()
    # Update a, b, c or d, depending on the contents of name.

  # Print the answer
\end{pythoncode}

The output requires four lines, so you will use four lines of code, each containing one
\pycode|print| statement. Be careful of alignment, as shown in the sample output.

One last thing before you complete this: the initialisation of our four counter variables
can be done in one line!

\begin{pythoncode}
  a, b, c, d = 0, 0, 0, 0
\end{pythoncode}

