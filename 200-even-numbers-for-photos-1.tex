
\questionheader{Even numbers for photos! (1) \stronghint}

\Question\ Some cultures have a strong preference that when a photograph of a group of
people is taken, that the group comprise an \emph{even} number of people. You, as a
wedding photographer who is about to work at a wedding that is steeped in one such
culture, have been given a list of groups who will be photographed, and you need to
scrutinise it to see if there are any, um, \emph{odd} problems that need to be fixed.

The list just contains the \emph{number} of people in each planned photo. You decide to
analyse it as follows: highlight the \emph{chains} of odd numbers so that you might engage
in some creative rearranging. For instance, the list might read: \[
  \mathbf{7}
  \quad 8 \quad 10 \quad 8
  \quad \mathbf{11} \quad \mathbf{13} \quad \mathbf{11} \quad \mathbf{19}
  \quad 4
  \quad \mathbf{9} \quad \mathbf{15}
  \quad 6 \quad 8
\]

In this list, you are interested in the bolded chains, which have lengths 1, 4 and 2,
respectively. You are also interested in:
\begin{itemize}
  \item the total number of such groups (in this case, 7); and
  \item the total number of people involved in these photos (in this case, 91).
\end{itemize}

Knowing these things won't solve all your problems, but it's a start.

\Input\ The first line is a positive integer $N$, which is the number of data points to
follow. The next $N$ lines each contain a positive integer $P_i$, each of which is the
number of people planned to be in a photograph.

\Output\ You will output the \emph{length of each chain} of odd-numbered photographs.
After this, leave a blank line, and output the number of photos containing an odd number
of people, and the total number of people in these photos, on one line as shown in the
samples.

\Sample

\minipagesthree{
  \sample{0.3}{13\\7\\8\\10\\8\\11\\13\\11\\19\\4\\9\\15\\6\\8}
         {0.3}{1\\4\\2\\\ \\7:91}
}{
  \sample{0.3}{6\\5\\3\\4\\8\\6\\3}
         {0.3}{2\\1\\\ \\3:14}
}{
  \sample{0.3}{7\\4\\6\\4\\2\\6\\4\\6}
         {0.3}{\ \\0:0}
}

\Explanation\ The first sample represents the example given in the question. The second
sample has odd chains $[5,3]$ and $[3]$, hence the output \texttt{2} and \texttt{1}. The
summary for the second sample is three groups totalling 14 people, hence \texttt{3:14}.
The third sample has no groups with an odd number of people, so there are no chains to
report. There is still a blank line before the summary \texttt{0:0}, though.

\Scratch\ Try creating and completing a trace table for the first sample before
continuing. A hint is given over the page.

\clearpage % ------------------------------------------------------------------ new page
\questionheadercontinued{Even numbers for photos! (1)}

We assume the code structure shown below.
\begin{pythoncode}
  N = int(IN.readline())
  # other initialisations...

  for i in range(N):
    # ...
\end{pythoncode}

What are we tracking in this question? We want the \emph{number of groups} and the
\emph{number of people}, which suggests variables \pycode|ngroups| and \pycode|npeople|.

Here is a trace table that you could try to complete.

\bigskip
\begin{center}
  \begin{tabular}{ccccc}
    \toprule
    ~~i~~ & size & ngroups & npeople & \emph{output} \\
    \midrule
    --    &      &         &         &               \\
    0     & 7    &         &         &               \\
    1     & 8    &         &         &               \\
    2     & 10   &         &         &               \\
    3     & 8    &         &         &               \\
    4     & 11   &         &         &               \\
    5     & 13   &         &         &               \\
    6     & 11   &         &         &               \\
    7     & 19   &         &         &               \\
    8     & 4    &         &         &               \\
    9     & 9    &         &         &               \\
    10    & 15   &         &         &               \\
    11    & 6    &         &         &               \\
    12    & 8    &         &         &               \\
    \bottomrule
  \end{tabular}
\end{center}
\bigskip

The idea of \pycode|ngroups| is to count the groups in the \emph{current chain}. It gets
reset when a chain finishes (i.e.~by encountering an even number). The idea of
\pycode|npeople| is to sum the people in relevant groups as you go through the data. This
variable does not get reset.

The column \emph{output} has not been seen before. The output from the program does not
all need to be produced at the end. In this program (and with this data), the summary
\texttt{7:91} will be produced at the end, but the rest of the output (\texttt{1 4 2} on
separate lines) will be produced as we go.

The completed table appears on the next page for you to check your work.


\clearpage % ------------------------------------------------------------------ new page
\questionheadercontinued{Even numbers for photos! (1)}

Here is the completed trace table.

\bigskip
\begin{center}
  \begin{tabular}{ccccc}
    \toprule
    ~~i~~ & size & ngroups & npeople & \emph{output} \\
    \midrule
    --    &      & 0       & 0       &               \\
    0     & 7    & 1       & 7       &               \\
    1     & 8    & 0       & 7       & \texttt{1}    \\
    2     & 10   & 0       & 7       &               \\
    3     & 8    & 0       & 7       &               \\
    4     & 11   & 1       & 11      &               \\
    5     & 13   & 2       & 24      &               \\
    6     & 11   & 3       & 35      &               \\
    7     & 19   & 4       & 54      &               \\
    8     & 4    & 0       & 54      & \texttt{4}    \\
    9     & 9    & 1       & 63      &               \\
    10    & 15   & 2       & 78      &               \\
    11    & 6    & 0       & 78      & \texttt{2}    \\
    12    & 8    & 0       & 78      &               \\
    \bottomrule
  \end{tabular}
\end{center}
\bigskip

From this table we can see the following.
\begin{enumerate}
  \item \pycode|ngroups| goes up each time we encounter an odd number. But when we
    encounter an even number, it goes back to zero, because we have reached the end of a
    chain.
  \item When the end of a chain is reached, this is a trigger for \pycode|ngroups| to be
    sent to the output.
\end{enumerate}

The sample data considered above is not in the middle of an odd chain when the data
terminates. If it were, you'd need to ensure that the size of the final chain was included
in the output. The second sample captures this occurrence.

\bigskip
\textbf{When coding your solution}, you need to know whether a number is even or odd. This
is simple enough, and it's the same in most programming languages. Try the following in
your console:
\begin{pythoncode}
  15 % 2        # 1
  16 % 2        # 0
  17 % 2        # 1
  18 % 2        # 0
\end{pythoncode}

The \pycode|% 2| means ``the remainder when you divide by 2'', which can only be zero or
one. Thus to determine whether the variable \pycode|size| is even or odd, you can write:
\begin{pythoncode}
  if size % 2 == 1:
    # size is odd
  else:
    # size is even
\end{pythoncode}

