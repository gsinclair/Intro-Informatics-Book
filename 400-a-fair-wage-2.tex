
\questionheader{\theproblemnumber\ A fair wage (2)}

\Question\ In the years since you completed \emph{A fair wage (1)}, your company has
grown and grown, so the quaint idea of reading in just six numbers and assigning each to
its own variable to each has pretty much faded from memory. You now have $N$ workers and
want to do the same test: check that the \emph{range} of wages is no greater than 10\% of
the maximum wage. Except in recent times you've found the 10\% measure to be a little too
inflexible. So now the target percentage $P$ will be included in your data.

\Input\ From \IN\ you will read:
\begin{itemize}
  \item the integer $N$ ($5 < N < 1000$), the number of employees at your company;
  \item the float $P$ ($1.0 < P < 30.0$), the percentage used to determine whether the
    wages are fair;
  \item $N$ lines, each containing a float representing an employee's weekly wage.
\end{itemize}

\Output\ To \OUT\ you will write three values:
\begin{itemize}
  \item The range of wages, rounded to two decimal placs
  \item The highest wage, rounded to two decimal placs
  \item \texttt{yes} or \texttt{no} according to whether your wages are fair
\end{itemize}

\Sample

\sample{0.2}{7\\8.5\\635.17\\622.25\\631.02\\631.02\\628.56\\599.75\\608.10}
       {0.2}{35.42\\635.17\\yes}

\Explanation\ In the sample, the range is about 5.58\% of the maximum wage, which is less
than the target percentage 8.5\%, so the wages are fair.

\Scratch\ You know how to find the \pycode/max/ or \pycode/min/ of a group of individual
numbers, but in this problem you need the maximum and minimum of a \emph{list} of numbers.
That's no problem. The Python functions are very flexible:
\begin{pythoncode} 
  max(3,7,5,1,2)           # -> 7
  max([3,7,1,5,2])         # -> 7
\end{pythoncode}

There are five arguments in line 1 and only one argument (a list) in line 2, but the
result is the same. \pycode/max/ can take any number of arguments and knows what to
do if it is passed a single list.

