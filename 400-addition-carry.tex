\questionheader{Addition carry}

\note{This exercise assumes some experience with lists.}

\Question\ When two numbers are added, carry can be involved. Given two numbers, perform
the addition using the normal primary-school algorithm and output the number of times a
carry was performed.

Both numbers provided in \IN\ will be 1--20 digits long. They will not necessarily be the
same length.

\Sample

\begin{minipage}[t]{0.2\textwidth}
  \ttfamily
  \textbf{IN} \\
  19526\\
  32287
\end{minipage} %
\begin{minipage}[t]{0.2\textwidth}
  \ttfamily
  \textbf{OUT} \\
  3
\end{minipage}

\vspace{6pt}
\begin{minipage}[t]{0.2\textwidth}
  \ttfamily
  \textbf{IN} \\
  232\\
  51
\end{minipage} %
\begin{minipage}[t]{0.2\textwidth}
  \ttfamily
  \textbf{OUT} \\
  0
\end{minipage}

\Explanation\ The first example involves the following additions:
\begin{itemize}
  \item $6+7$ (generating a carry)
  \item $2+8+1$ (generating a carry)
  \item $5+2+1$
  \item $9+2$ (generating a carry)
  \item $1+3+1$
\end{itemize}

There are three carries.

The second example clearly involves no carries.

\Scratch\ To solve this problem we need to look at each number as a list of (numeric)
digits instead of a number \emph{per se}. So we read in the two numbers as strings and use
some special code that extracts the digits.

\begin{pythoncode}
  S1 = IN.readline()            # "19526"
  S2 = IN.readline()            # "2287"
  A = [int(x) for x in S1]      # [1, 9, 5, 2, 6]
  B = [int(x) for x in S2]      # [2, 2, 8, 7]
\end{pythoncode}

The code \pycode|for x in S1| iterates through each character \pycode|'1', '9', '5', '2',
'6'| in \pycode|S1|, and of course \pycode|int(x)| converts each of those to an integer.
Enclosing that code in \pycode|[...]| collects the generated values in a list. This kind
of code, called a \emph{list comprehension}, is a kind of superpower that Python has,
although it may be a bit daunting at first.

For the algorithm we are presently implementing, we would really like our two digit lists
to be in reverse order. The following code will do it.

\begin{pythoncode} 
  S1.reverse()                  # S1 is now [6, 2, 5, 9, 1]
  S2.reverse()                  # S2 is now [7, 8, 2, 2]
\end{pythoncode}

And now the digits line up and you can move \emph{forward} through the two reversed lists.

\Solution The rest of the code for this problem is an exercise for you.

