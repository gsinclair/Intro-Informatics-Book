
\chaptermacro{Introduction}

The exercises in this document will enable you to progress from absolute informatics
novice to a level of confidence sufficient to tackle elementary problems on
ORAC\footnote{ORAC is an online judge found at orac2.info. Many other online judges exist,
such as dmoj.ca, hackerrank.com, and so on. ORAC has excellent problems, but its
input/output system is a pain, especially for beginners, and its entry-level problems,
while very easy, are still too hard for complete beginners.}, such as \emph{Cute Numbers},
\emph{Ladybugs}, and \emph{Drought}.

These exercises are self-contained and do not involve submitting code to a website. You
do, however, need to set up your Python coding environment a little. This up-front
inconvenience pays off very quickly, though, as you can:
\begin{itemize}
  \item Solve all problems in one file.
  \item Run your code interactively, or test it against known data, or have it judged
    against secret data, all with ease.
\end{itemize}

\emph{Describe the PLC Informatics system, or refer to Appendix A or whatever...}

The problems in Series 100 are generally original, though some inspiration naturally comes
from easy problems I've seen before. They have only a modicum of context, unlike later
informatics problems that require you to read paragraphs of text to discern exactly what
the problem is. Some problems will even give you some or all of the code required to solve
them! The aim is for you to experience a minimum of friction in the early part of your
learning.

\emph{Idea: mention here that we want to get right into it, but if you are reading this
with some interest but wondering what it's actually all about, please see Appendix A
``What is informatics anyway?''}

For ease of navigating the document, problems appear one per page. Please turn over and
begin.

\section*{Problems to develop}
\begin{itemize}
  \item Input \texttt{940 5 . + 30 . -5 . + 100 . + 12 . - 4} representing a starting
    value and a series of additions and subtractions. Output \texttt{940 . 970 . 965 .
    1065 . 1077 . 1073} representing the starting value, the intermediate values, and the
    answer. Story can be something like profits and losses each week, and wanting to see
    that the business is afloat. Perhaps if an intermediate value is negative, the output
    can include a warning alongside it.
  \item Sarah has taken up jogging. She starts at $D$ metres per day and increases her
    distance by $I$ metres each day. How many days until she is running 10 kilometres per
    day?
  \item Siobhan has taken up jogging. She starts at $D$ metres per day and increases her
    distance by $I$ metres each day until she reaches her limit of $M$ metres per day.
    Find the total distance she covers in $N$ days.
  \item Find the length of a Collatz chain starting at a given number.
  \item Find which starting number between $A$ and $B$ produces the longest Collatz chain.
  \item Four people (Albert, Betty, Charlie, Dolly) play several games of Scrabble. They
    know who won each game but have lost track of their overall tally. Input: $N$ lines of
    names. Output: \texttt{Albert: 48 . Betty: 37} etc.
  \item Cute numbers
\end{itemize}
