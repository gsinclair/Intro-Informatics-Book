
\chaptermacro{Introduction}

If you are a complete beginner at Informatics, then the examples and exercises in this
book aim to bring you up to the level of ``relative beginner''. This may seem like a
modest aim, but it is not. It is like setting a car in motion: it takes a lot of engine
power to get a stationary car moving, but not so much to keep it moving. You will need
considerable brainpower, focus and effort to take the first steps in Informatics, but once
you have a reasonable idea of what's going on, you will be able to direct your own
progress. There will still be a lot to learn, but you won't be weighed down by confusion
over the basics.

The rest of this introduction is as brief as possible so you can get started!

\paragraph{What is Informatics?} At the simplest level, it is a form of computer programming
dealing with \emph{information}. A problem is stated, and you write a program that will
solve that problem \emph{given some information}. Here are some examples from this book:
\begin{itemize}
  \item Given the heights of four people, determine the height of the tallest person.
  \item Given an arbitrary number of quantity-price pairs, determine the total amount
    spent.
  \item Given a grid of squares containing cash amounts, find the location that has the
    highest total in the surrounding squares.
\end{itemize}

These are all simple problems because this is a book that focuses on solving simple
problems. When you enter Informatics competitions you will be exposed to much harder
problems, akin to finding the shortest path through a maze or solving a Rubik's Cube.

\paragraph{How to use this book} Each chapter contains some teaching material, one or two
fully worked examples, and some exercises. Read the teaching material carefully and try
out its code examples. All exercises, including the worked examples, are numbered, and you
will use the \textbf{learninformatics} Python environment to complete the exercises and
have them confirmed as correct.

\paragraph{Setting up the learninformatics environment} Full instructions follow, but the
information is best conveyed by a demonstration. See the video at \futurereference{YouTube
link}.

While it is possible to use a Python environment on your own computer,%
\footnote{See \futurereference{Appendix A} for details.}
we prefer here to outline the simpler approach of using the website \textbf{replit.com}.

\begin{enumerate}
  \item Log in to your account on replit.com, or create one if you don't already have one.
    If it asks for your preferred languages, chooose Python, Ruby and Clojure.
  \item Create a new repl%
\footnote{repl is an initialism for read-eval-print-loop. It really means an interactive
  programming environment.}
    in the Python language. Title it \emph{LearnInformatics}.
  \item Create a file in your repl called \texttt{learninformatics.py}.
  \item Go to the URL footnoted,%
\footnote{\texttt{https://raw.githubusercontent.com/gsinclair/learninformatics/master/learninformatics.py}}
    copy all of the text, and paste it into the file you just created.
  \item In \texttt{main.py}, type: \pycode|import learninformatics as l|
  \item Run your code with the green Run button.
  \item In the \emph{Console}, type \texttt{l.update()} and hit Enter/Return. This will
    create the file \texttt{DATA.txt} in your project. You can run \texttt{l.update()} at
    any time to ensure you have the latest data.
\end{enumerate}

You should now be ready to go.

\paragraph{Using the learninformatics environment} You'll get the hang of it by following
instructions later on, but here is a summary. Say you are working on exercise 205. You
will add the following to \texttt{main.py}:%
\footnote{Your code will accrete many small sections like this: \pycode|ex101|,
  \pycode|ex102|, ...}
\begin{pythoncode}
  def ex205(IN, OUT):
    # Get input data from IN.
    # Solve the problem.
    # Write the answer to OUT.
\end{pythoncode}

When you \emph{Run} your code, nothing will happen, because all you have is a bunch of
definitions. But in the \emph{Console} you can type \pycode|l.samples(205)| and it will
execute your code in \pycode|ex205| and provide it with the sample input data (specified
in the exercise) and check that the code produces the correct output. When you are getting
correct results for the sample data, you will type \pycode|l.judge(205)| to see if your
code works for a wider variety of inputs.

This might sound complicated, but it's a much easier workflow than the normal Informatics
practice in which each problem is solved in its own file and submitted to a website for
marking.

Here are example commands and what they do:
\begin{center}
  \begin{tabular}{lp{10cm}}
    \toprule
    \textbf{Command} & \textbf{Action} \\
    \midrule
    \texttt{l.run(101)}     & Run your function \texttt{ex101} interactively, taking
                              input from the keyboard and sending output to the
                              screen.\\[3pt]
    \texttt{l.run(example)} & Run your function \texttt{example} interactively.\\[3pt]
    \texttt{l.samples(205)} & Run your function \texttt{ex205}, using the sample data
                              set(s) as input and checking that your code generates
                              the correct output.\\[3pt]
    \texttt{l.judge(205)}   & Run your function \texttt{ex205} multiple times, using a
                              variety of (secret) data sets and checking that your code
                              generates the correct output in all cases.\\
    \bottomrule
  \end{tabular}
\end{center}

%%% \section*{Old introduction text}
%%% 
%%% The exercises in this document will enable you to progress from absolute informatics
%%% novice to a level of confidence sufficient to tackle elementary problems on
%%% ORAC\footnote{ORAC is an online judge found at orac2.info. Many other online judges exist,
%%% such as dmoj.ca, hackerrank.com, and so on. ORAC has excellent problems, but its
%%% input/output system is a pain, especially for beginners, and its entry-level problems,
%%% while very easy, are still too hard for complete beginners.}, such as \emph{Cute Numbers},
%%% \emph{Ladybugs}, and \emph{Drought}.
%%% 
%%% These exercises are self-contained and do not involve submitting code to a website. You
%%% do, however, need to set up your Python coding environment a little. This up-front
%%% inconvenience pays off very quickly, though, as you can:
%%% \begin{itemize}
%%%   \item Solve all problems in one file.
%%%   \item Run your code interactively, or test it against known data, or have it judged
%%%     against secret data, all with ease.
%%% \end{itemize}
%%% 
%%% \emph{Describe the PLC Informatics system, or refer to Appendix A or whatever...}
%%% 
%%% The problems in Series 100 are generally original, though some inspiration naturally comes
%%% from easy problems I've seen before. They have only a modicum of context, unlike later
%%% informatics problems that require you to read paragraphs of text to discern exactly what
%%% the problem is. Some problems will even give you some or all of the code required to solve
%%% them! The aim is for you to experience a minimum of friction in the early part of your
%%% learning.
%%% 
%%% \emph{Idea: mention here that we want to get right into it, but if you are reading this
%%% with some interest but wondering what it's actually all about, please see Appendix A
%%% ``What is informatics anyway?''}
%%% 
%%% For ease of navigating the document, problems appear one per page. Please turn over and
%%% begin.

\section*{Problems to develop}
\begin{itemize}
  \item Input \texttt{940 5 . + 30 . -5 . + 100 . + 12 . - 4} representing a starting
    value and a series of additions and subtractions. Output \texttt{940 . 970 . 965 .
    1065 . 1077 . 1073} representing the starting value, the intermediate values, and the
    answer. Story can be something like profits and losses each week, and wanting to see
    that the business is afloat. Perhaps if an intermediate value is negative, the output
    can include a warning alongside it.
  \item Find the length of a Collatz chain starting at a given number.
  \item Find which starting number between $A$ and $B$ produces the longest Collatz chain.
  \item Cute numbers
\end{itemize}
