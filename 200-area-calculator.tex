\questionheader{\theproblemnumber\ Area calculator}

\Question\ As you pore over your Maths homework, you tire of the seemingly endless and
pointless area calculations you have to do. All the circles, semicircles, rectangles,
squares, parallelograms and triangles demand so much attention, and you only have so much
to give. But if you write a program to do the calculations for you, you'll get this
wretched work done and have time for a camomile tea while watching \emph{Rick \& Morty}
before bed.

\Input

The input contains any number of lines, which represent the shape names and the values
required to calculate them. For circles and semicircles you get one value: the radius. For
rectangles, parallelograms and triangles, you get base and (perpendicular) height. For
squares, you get the length of one side. The last line of input is simply \texttt{stop}.

Your input will contain ten area calculations or fewer.

\Output\ The output contains several lines, each of which is the result of an area
calculation, which must be a float rounded to three decimal places.

\Sample

\sample{0.25}{circle\\4.6\\triangle\\12\\5\\parallelogram\\19\\4.5\\
             square\\19\\rectangle\\7.2\\3.6\\stop}
       {0.2}{14.451\\30.0\\85.5\\361.0\\25.92}

\Scratch\ This problem differs from the earlier ones in two ways:
\begin{enumerate}
  \item We are \emph{not} told at the beginning how many lines there will be.
  \item Depending on the shape, we may need to read one line or two.
\end{enumerate}

Therefore, you deserve some help with this problem.

To solve \#1, we want to loop \emph{endlessly} and break out of the loop when we read
\texttt{stop}. This can be done with the following code structure:
\begin{pythoncode}
  while True:
    # ...
    if (condition):
      break
    # ...
\end{pythoncode}

The \pycode|while True| part creates the endless loop, and the \pycode|break| statement
gets out of it. Naturally, the \emph{condition} for us to break out is having read the
word \texttt{stop}.

Solving \#2 is not difficult and you probably don't need help with it, but just in case,
there is a hint on the next page.

\clearpage

\Hint

Here is the structure of the code, excluding the function header.

\begin{pythoncode}
  while True:
    shape = IN.readline().strip()
    if shape == 'stop':
      break
    elif shape == 'circle':
      radius = float(IN.readline())
      area   = 3.141592654 * radius * radius
      print(round(area,3), file=OUT)
    elif shape == 'rectangle':
      base   = float(IN.readline())
      height = float(IN.readline())
      # ...
    # etc.
\end{pythoncode}

