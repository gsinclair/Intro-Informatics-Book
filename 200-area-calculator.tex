\questionheader{\theproblemnumber\ Area calculator}

\Question\ As you pore over your Maths homework, you tire of the seemingly endless and
pointless area calculations you have to do. All the circles, semicircles, rectangles,
squares, parallelograms and triangles demand so much attention, and you only have so much
to give. But if you write a program to do the calculations for you, you'll get this
wretched work done and have time for a camomile tea while watching \emph{Rick \& Morty}
before bed.

\Input

The input contains any number of lines, which represent the shape names and the values
required to calculate them. For circles and semicircles you get one value: the radius. For
rectangles, parallelograms and triangles, you get base and (perpendicular) height. For
squares, you get the length of one side. The last line of input is simply \texttt{stop}.

Your input will contain ten area calculations or fewer.

\Output\ The output contains several lines, each of which is the result of an area
calculation, which must be a float rounded to three decimal places.%
\footnote{See \futurereference{Appendix ...} for instructions on rounding a float.}

\Sample

\sample{0.2}{circle\\4.6\\triangle\\12\\5\\parallelogram\\19\\4.5\\
             square\\19\\rectangle\\7.2\\3.6\\stop}
       {0.2}{14.451\\30.000\\85.500\\361.000\\25.920}

