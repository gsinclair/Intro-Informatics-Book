
\questionheader{\theproblemnumber\ High-wire walk}

\Question\ Philippe the high-wire artist is planning his next audacious high-wire walk in
New York City. He wants to walk over as many buildings as possible, but at a prescribed
height.

Fortunately, he has a list of consecutive building heights and he wants you to assist in
finding the best option for his walk.

For example, say there are 12 buildings in the list and Philippe wants to walk at a height
of 80 metres. Then the possible options are highlighted below.\[
  100 \quad
  \mathbf{50} \quad \mathbf{60} \quad
  90 \quad 110 \quad
  \mathbf{20} \quad \mathbf{50} \quad \mathbf{40} \quad \mathbf{70} \quad
  130 \quad 110 \quad 70
\]

Note that the 70 at the end is not highlighted because there is no tall building to the
right of it that can be used as an anchor.

The best option for Philippe in this scenario is to walk over four buildings.

\Input\ The first two lines contain the integers $N$ and $H$, which represent the number
of buildings in the list and the height at which Philippe wants to walk.

Following that are $N$ positive integers representing the heights $B_i$ of the buildings.

You are guaranteed that $5 \le N \le 1000$, $10 \le H \le 500$ and $10 \le B_i \le 1000$.

\Output\ You will output one non-negative integer: the largest number of buildings over
which Philippe can walk.

\Sample

\sample{0.4}{12\\80\\100\\50\\60\\90\\110\\20\\50\\40\\70\\130\\110\\70}
       {0.4}{4}

\Explanation\ The sample data matches the scenario given in the example.
