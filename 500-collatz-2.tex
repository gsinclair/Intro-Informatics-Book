
\questionheader{\theproblemnumber\ Collatz (2)}

\Question\ An example of a \emph{Collatz chain} is as follows: \[
  7 \to 22 \to 11 \to 34 \to 17 \to 52 \to 26 \to 13 \to 40 \to 20 \to 10 \to 5 \to 16 \to
  8 \to 4 \to 2 \to 1.
\]

As stated earlier in \emph{Collatz (1)}, this is the rule: if a term $T_n$ is even, then
$T_{n+1}$ is set to $\frac12 T_n$. If instead $T_n$ is odd, then $T_{n+1}$ is set to
$3\cdot T_n + 1$. If and when 1 is reached, the chain terminates. As far as we know, all
chains terminate.

In this problem, you are given two positive integers $A$ and $B$, and you find the length
of the longest chain produced.

\Input\ One line containing two positive integers $A$ and $B$. Both values are between 2
and 10\,000, inclusive.

\Output\ A single positive integer, being the length of the longest chain encountered when
using all starting values between $A$ and $B$.

\Sample

\sample{0.2}{10 16}{0.2}{18}

\Explanation\ The table below shows the chain lengths for all starting terms 7--16.

\begin{center}
  \begin{tabular}{cc}
    \toprule
    Starting term & Chain length \\
    \midrule
    10            & 7            \\
    11            & 15           \\
    12            & 10           \\
    13            & 10           \\
    14            & 18           \\
    15            & 18           \\
    16            & 5            \\
    \bottomrule
  \end{tabular}
\end{center}

The longest chain produced here is 18, hence the answer.

The fact that there were two chains of length 18 is unimportant.
