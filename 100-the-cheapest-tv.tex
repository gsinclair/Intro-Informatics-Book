\questionheader{\theproblemnumber\ The cheapest TV}

\Question\ You walk the aisles of Bing Lee, looking for a new TV. You narrow it down to
five choices and decide to go with the cheapest. How much will you pay?

Your program will read five integers (representing prices in dollars) from \IN\ and write
the smallest number to \OUT.

\Sample

\sample{0.2}{499\\565\\325\\400\\717}
       {0.2}{325}

\Scratch\ If you have learned the lessons of the previous exercise well, you will complete
this one in no time. The running instructions are the same as before except the problem
name is \problemtagtt.

\Afterword\ Run your code as before. Brief instructions are included here for the last time:
\begin{itemize}
    \item \code{p.run(\problemtag)} to run it interactively using keyboard and screen.
    \item \code{p.test('\problemtag', \problemtag)} to run it against some test data and
      get detailed information if there is an error.
    \item \code{p.judge('\problemtag', \problemtag)} to run it against some judging data
      and get a token if you are successful.
\end{itemize}

